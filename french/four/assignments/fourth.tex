\documentclass[12pt,a4paper]{article}

\usepackage[cyr]{aeguill}
\usepackage[frenchb]{babel}
\usepackage[utf8]{inputenc}
\usepackage[T1]{fontenc}
\usepackage{titling}
\usepackage{fullpage}

\setlength{\droptitle}{-10em}

\begin{document}

\title{\og L'exploration de l'espace est une perte de temps et de
ressources. \fg~Discutez.}
\author{Isabell Long (12945093)}
\maketitle

L'exploration de l'espace est un concept encore assez bizarre pour le
plupart de gems dans ce monde moderne, mais très important pour notre
avenir et ceux qui travaillent dessus. L'exploration de l'espace a
commencé il y a cinquante ans avec Sputnik, un satellite soviétique
lancé en 1957, et a progressé à une vitesse assez grande depuis dans
beaucoup de pays développés.\\

Il y a desormais, comme avec tout, des personnes qui pensent que cette
exploration s'agit d'une perte de temps et d'un gaspillage de
ressources --- des sceptiques. Elles donnent des arguments assez
lucides, comme si l'on se concentrait sur la Terre et améliorer notre
qualité de vie ici, on pourrait la sauver de la rechauffement
climatique. Il est important quand même de découvrir où est-ce que,
peut-être, les êtres humains viveront quand le rechauffement
climatique de la Terre nous engloutit.\\

Ces recherches n'étaient pas lancées qu'avec le but de découvrir où
est-ce que l'Homme pourrait se mettre dans une centaine d'années avant
qu'on soit tué par la Terre qui explode dû à la rechauffement
climatique ou d'autres catastrophes, mais aussi à découvrir ce qui
nous entoure dans un sens général, pour aider à répondre aux problemes
comme les inondations et les feux. Ceci est important parce que
l'exploration de l'espace ne concerne pas que les planètes à grande
distance comme le Jupiter ou le Saturne, mais aussi les étoiles comme
Alnitak et, bien sur groupé avec les étoiles : le Soleil. Le Soleil
n'est pas toujour considéré comme une étoile ou un planète, mais une
présence constante dans le ciel durant les heures généralement où l'on
travail.\\

Les pays qui ont commencé ces recherches, par exemple la Russie dans
les années '50, suivi assez vite une quatorzaine d'années après par
les États-Unis avec la création de NASA, n'étaient pas toujours les
pays développés dans le sens qu'on comprend maintenant. Il y a
beaucoup d'argent dans l'exploration de l'espace, investi par beaucoup
d'autres pays qui n'ont pas toujours les ressources pour lancer une
mission eux-mêmes mais qui assistent aux missions en fournissant les
astronautes. Par exemple, Samantha Cristoforetti vient de l'Italie et
fait à présent partie d'une mission appelée \textit{Expedition 42}
sur la Station Spatiale Internationale qui orbite notre Terre chaque
jour.\\

L'éxploration de l'espace est aussi très important pour les jeunes.
Pas seulement pour leur avenir en existant, mais en avancent leurs
rêves et leur faire pense au fait qu'on soit qu'un très petit point du
monde éxterieur. Beaucoup de personnes de n'importe quel age ont été
interessées quand même par l'astronef Rosetta qui a atterri sur la
comète \textit{67P/Churyumov--Gerasimenko} en novembre. Il nous
permettra de découvrir de quoi les comètes sont composées et qui
aidera à la recherche très connu : s'il y a de l'eau sur d'autres
objets dans le système solaire que la notre.\\

Pour conclure, je ne suis pas d'accord avec le point de vue que
l'exploration de l'espace soit une perte de temps et de
ressources. Si cela était le cas, pourquoi les gouvernements pour
plusieurs dizaines d'années auraient-ils investi pour créer des
institutions comme le \textit{NASA} ou l'\textit{Agence Spatiale
Européene}, qui ont pour but d'élargir nos connaissances dans cet
domaine.

\end{document}
