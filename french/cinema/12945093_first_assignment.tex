\documentclass[12pt,a4paper]{article}

\usepackage[utf8]{inputenc}
\usepackage[T1]{fontenc}
\usepackage[numbers]{natbib}
\usepackage{titling}
\usepackage{fullpage}
\usepackage{url}
\usepackage{breakurl}
\usepackage[breaklinks]{hyperref}

\setlength{\droptitle}{-10em}

\begin{document}

\title{Compare and contrast the cinematic representation of desire in
       \textsl{Le Mépris} and \textsl{La Vie d’Adèle}.}
\author{Isabell Long (12945093)}
\maketitle

% Introduce each of the films.
% What is desire?
\textit{Contempt}, or in French \textit{Le Mépris}, was released in
1963 and is a film directed by Jean-Luc Godard that concerns itself
with heartbreak and distintegration of a relationship, based around a
failing film director within the film itself. \textit{Blue is the
Warmest Colour}, or in French \textit{La Vie d'Adèle: Chapitres 1 et
2}, is, in contrast, a modern film released in 2013 and directed by
Abdellatif Kechiche. It touches on the subjects of teenage
coming-of-age, sexuality and romance. In this essay, I will
investigate the cinematic representation of desire. Desire is an
interesting word here because these two films seem to involve it from
two different standpoints---one from loss of affection and a stunted
relationship, and the other from a journey of desperate teenage
self-discovery which is very emotionally raw and which every teenager
can relate to in some way.\\

% Social context of 1963 and 2013. How had times changed?
To be able to fully analyse desire as it is portrayed in both films,
its connotations and how it might have affected viewers---both us and
the public---at the time it was released and now, upon first or second
watching, it is important to understand the social context of the
times in which the films were made and released. In the 1960s, France
was rebuilding itself after the War, whereas in 2013 it was modern and
generally progressive. It could be said that the breaking down of the
relationship \textit{Le Mépris} is symbolic of French society's loss
of so many figures, or people changing beyond recognition, relying not
on emotions but physical manifestations (``I like your hair; I like
your breasts.'') because they were easier than thinking about your own
or others' inner feelings and thereby causing fractures. The
rebuilding of the relationship in \textit{Blue is the Warmest Colour}
could be said not just to represent mindless teenage desire, but
France's progression and liberty---the admission of same-sex
relationships in themselves---but society's hatred of anything
different (Adèle's school friends' reactions to her going to a gay
bar, for example), and that deteriorating relationship symbolic of
France struggling to stay quite so liberal in these modern times with
fresh issues not thought about before, such as immigration and
terrorism.\\

The titles of both of the films, particularly the second, do not seem
at first to be particularly related to desire. \textit{Contempt}, for
example, does not at all seem ``desireable'' as it were, as
``contempt'' is most certainly not desire---one would not desire to be
seen with contempt by another person, least of all one they had
relations with in the past or currently, the latter of which the film
portrays. The title of \textit{Blue is the Warmest Colour} does not
scream desire at the reader either, but intrigue, especially if you do
not see the trailer or poster for the film or read anything about it
beforehand. In French, the title could be interpreted to be even more
dull: Adèle's life, but one could not so much as wonder what about it
is so special, and decline to watch the film.\\

% Sex scenes.
Desire can take many forms: desire of objects, desire for strength,
desire for wealth, desire for tolerance~(\citeauthor{stanford-desire},
Stanford Encyclopedia of Philosophy, 2015). Here, in both films, we
discuss desire to be normal, desire to have a relationship, and the
desire to be understood and accepted however we are as people. Desire
of a person can in many cases lead to a sexual
relationship~(\citeauthor{desire-evolution}, 1994), and we see this
particularly in the second film with the girls' ravishing of each
other. Conversely, the breaking down of the relationship between
Camille and Paul in \textit{Contempt} means that they hardly touch
each other, and this lack of desire portrayed by the lack of sex is
also mirrored at the end of \textit{Blue is the Warmest Colour} when
Adèle suspects---correctly---that Emma is pulling away from her after
she makes advances but is pushed away with an excuse that she knows
not to be true because of the timing: ``I'm on my period''. Both of
the films could be said to be channelling or anticipating the viewer's
desires, though, with their teasing and in the second film's case the
prolific sex scenes, the longest of which is ten minutes long and not
something that people are normally used to seeing portrayed so
graphically even in recent years. The portrayal of Camille in
\textit{Contempt} is very understated as she tempts Paul, but then
there is a naked scene where we see Camille---played by the French sex
symbol of the 1960s Brigitte Bardot---fully naked for a long period
talking to Paul. Some may deem this excessive, and there is
evidence~(British Film Institute,
\citeyear{contempt-nakedness-addition}, and in an interview on the
DVD) to suggest that it was added at a later date to appease the
director who was worried about making the general public likely to
watch the film because they sexualised the main character.\\

% Cinematic features. Types of shots, words, lighting, etc.
In terms of the cinematic features rather than the cultural ones,
there are many interesting moments throughout both films, in relation
to shot types and language used. \textit{Le Mépris} is particular in
that it contains shots of another film which the protagonist Paul is
in the process of directing as part of the story. For the first time,
Godard had the chance and budget to use the CinemaScope and
Technicolor technology and hire good actors, as the French cinema
business was booming in the wake of people returning from war and
wanting to distract themselves from their nightmares, and wartime bans
on specfic types of films having been lifted. This made the film very
experimental: Godard tends to use tracking shots which follow the
speaking character around the room, and some could say he overuses the
technique of panning---hardly ever showing the two characters in the
same room in the same shot. An example of this is when Camille shouts
from the bathroom that Paul does not love him anymore, and they
converse shouting at each other instead of entering the same room,
while Paul tries to read a book. When they do enter the same
room---the bathroom---she quickly covers herself up and leaves,
perpetuating the feeling of disconnection between them and leaving the
viewer very perplexed as to what is happening between the two central
characters and whether they are going to last much longer.\\

In \textit{La Vie d'Adèle}, the camerawork is very efficient. From
almost the first moment, the viewer does not know what he or she even
expects from the film as we see Adèle walking aimlessly and of course
smoking, perpetuating a French stereotype. We see that she is quite
young---supposedly sixteen at the time---and we wonder what she will
do with her life. Throughout the film, there is a predominance of
close-up shots, whether they be on Adèle masturbating over a fantasy
of Emma, eating, smoking, drinking or more often than not, crying or
seeming in the depths of thought or despair. Other than that, the film
portrays the girls very closely, with all of the rifts between Adèle
and her friends in the playground seeming very realistic---just what
teenagers would say about their friends when they discover they are
not what they thought they were and do not fit in with social norms.
Adèle protests her innocence as it were, somewhat falsely, and then
her best male friend lets her down and she storms out to Emma who is
waiting for her. They walk off hand in hand and that is quite a
pivotal moment in Adèle's journey of self-discovery. After that, there
is a scene where Emma draws Adèle and class divides rear their head,
but are overshadowed by how young and vulnerable Adèle looks in the
sketch. They laugh it off as Emma's inexperience at sketching
people---``I'm not used to sketching faces''---but when Emma leaves
there is a close shot of Adèle appearing to look puzzled and seeming
to wonder if she is just dreaming about ever feeling secure in her own
skin. It is not all about lesbianism, though, as Adèle dances with
someone from work later on in the film, when she has moved on from
school to become a teacher and is in fact living with Emma. This dance
is very sensual, and music becomes a big part as well as the camera
watching through the window at Adèle as she removes clothes, as if the
viewer is watching something forbidden, and ends up kissing the male
colleague. This segues nicely but unexpectedly into the hectic and
horrific fight scene which shows them both on the edge of despair, no
feelings on Emma's part. Emma appears very short and emotionless, not
really crying, mostly wanting to never see Adèle again. One wonders
why Emma would banish Adèle with no explanation if she herself had not
wanted to end the relationship and was just looking for an excuse.
This, although unsaid, seems to transpire to be the case.
Interestingly, the removed nature of the Adèle's dancing scene
contrasts heavily with the very intense and intimate nature of the sex
scenes, with nothing left to the imagination, despite the actresses
use of prosthetic genitalia. Most damningly, the author of the
original graphic novel, Julie Maroh, said that 2013 the film was
missing ``real lesbians'' after attacking its portrayal of her story
by it being shot from a man's perspective and Kenniche teasing
audiences with gratuitous sex. She also compared these scenes to
pornography~(Guardian News and Media, 2013\nocite{author-interview}),
especially with the perfectly formed prosthetics.\\

% Why was La Vie d'Adèle so acclaimed? Was it `real'?
\textit{Blue is the Warmest Colour} was very highly acclaimed,
narrowly missing out on a Golden Globe award having been nominated for
`best foreign language film' in 2014. Instead, the film, its director
and main actresses were famed at the Cannes film festival, where they
won the 2013 Palme d'Or\nocite{palme-dor}. Nevertheless, the film
was shrouded in controversy as some claimed it wasn't a true portrayal
and that the actresses were mistreated, having to shoot scenes
tens of times to get the perfect shots~(\citeauthor{reshoot-scenes},
2013). In some cases, however, this made the scenes all the more real
because the actresses were physically exhausted and\slash or
emotionally drained by the end---in the fight scene for example, where
Emma kicks Adèle out because she cheated on her, Adèle was really
crying (also~\citeauthor{reshoot-scenes}, 2013).\\

% How do both of the films finally represent desire, and why in these ways?
In conclusion, desire is portrayed very differently in both
\textit{Contempt} and \textit{Blue is the Warmest Colour}. The one the
viewer prefers is down to personal preference---the romantic tragedy,
or the almost-pornographic lesbian coming-of-age love story that
eventually ends in a journey of self-discovery far beyond the
relationship itself. Desire is not all about sex, it is about
enrichment, love, and loss, which both films portray very convincingly
and which is very relateable by everyone, whoever you are, as it is a
central tenet of society.

\bibliographystyle{plainnat}
\bibliography{first_assignment}

\end{document}
