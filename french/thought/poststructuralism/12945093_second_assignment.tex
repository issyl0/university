\documentclass[12pt,a4paper]{article}

\usepackage[utf8]{inputenc}
\usepackage[T1]{fontenc}
\usepackage[numbers]{natbib}
\usepackage{titling}
\usepackage{fullpage}
\usepackage{url}
\usepackage{breakurl}
\usepackage[breaklinks]{hyperref}

\setlength{\droptitle}{-10em}

\begin{document}

\title{How is the concept of capitalism, as developed by Deleuze and
       Guattari, relevant to a discussion of the self and the other?}
\author{Isabell Long (12945093)}
\maketitle

Capitalism is defined in the Oxford English Dictionary as ``[a]n
economic and political system in which a country's trade and industry
are controlled by private owners for profit, rather than by the
state.'' Deleuze and Guattari were a pair of French philosophers who,
in 1972, released a book entitled \textit{«~L'Anti-Oedipe~:
Capitalisme et Schizophrénie~»} (in English \textit{``Anti-Oedipus:
Capitalism and Schizophrenia''}). This was a multi-faceted critique of
many parts of previous philosophers' and psychoanalysts' works,
notably Karl Marx and Sigmund Freud. The concept of the self and the
other is also known as alterity, and concentrates on the differences
between one's own being, their habits, thoughts, perceptions, feelings
and emotions, and another being's same set of viewpoints on life. In
this essay I will combine these three themes and analyse their effects
on society, with particular reference to capitalism's relationship to
the principle of the self and the other.\\

Many of us in the Western, or wealthy, world are, sadly, very selfish.
In the book \textit{23 Things They Don't Tell You About Capitalism}
(Chang, 2010) dispels some myths but allows some others, notably that
economists consider society, that on which capitalism was born,
selfish, thinking only of themselves, their business profits, such
that the people that make up society are mostly ``tunnel-visioned,
self-seeking robots''. This contrasts incredibly strongly with the
self and the other approach to life, given that the self and the other
approach considers us, one person, and them, another person, for want
of a better set of words, equals: we all have the same feelings and
perceptions on a broad, underlying level, even if day to day they are
not the same---I as a person may want to buy a PlayStation 4, while
the person next to me might want to buy a doll's house for their
child. The principle of capitalism in its forms these days is that
``people won't do good unless they are paid for it'' (also Chang,
2010). Of course, this is a generalisation of all of the (again,
wealthy) population: a surprising number of people do volunteer their
time for good causes, for example charity work or sports events purely
benevolently, thereby requesting or, indeed, receiving nothing back
for it. In the United Kingdom, there are at present 170,843 volunteers
registered on volunteering opportunity website \textsl{www.do-it.org},
and many more unaccounted for, as such, who choose not to publicise
their volunteering, or sites like that do not cover their particular
type of volunteering activity.\\

The allusion in the previous paragraph brings me on nicely to the
distinction between capitalism and consumerism. In Deleuze and
Guattari's day, the world was not quite as consumerist as it is
now---intent on buying anything and everything, even though they don't
need it, some may say in general even blind to the plight of those
less fortunate---but it was becoming very capitalist: companies were
privatising, coming out of state control and, the big thing: being
forced to turn a profit in order to survive. Therefore, it can be said
that capitalism breeds consumerism, with consumerism then breeding more
capitalism because business people---even startups, these days,
especially in the technology industry---see the general public buying
more and more things and want to get in on the action, so build more
business selling the same or similar things, offering the same or
similar services, and as such people spend more money.\\

Inversely, capitalism isn't all that bad in terms of boosting the
economy: more non-State-funded business create more jobs for the
common people, which in turn gives them money to go and spend in
different shops, leading to constant regrowth of profits and markets.
However this may seem, whether it is good or bad in the reader's point
of view, Deleuze and Guattari made a point of saying that all
desire---whether it be the desire to work, the desire to love, the
desire to feel like a true member of society (which you do by doing
the above and more---particularly working and having that unspoken
social contract in your life)---is negative, as per Freudian and
Platonian philosophy, and built upon by societal machinery (Bernico,
M.\, 2012). In that they mean that man (\textit{l'Homme}) doesn't
really desire anything, but desire and wants are social constructs
that end up repressing people as it is all they think about.
Essentially, one can never be truly happy, and that is what
capitalism, and definitely consumerism, relies and thrives upon.\\

Therefore, if one can never truly be happy because of societal
pressures and contracts that must not be broken for fear of becoming
an outcast---shopping and work, to name just two quite shallow yet
important ones in our lives today---it is worth analysing at what
point the other in ``the self and the other'' is thought of, and why
we continue to perceive ourselves as individuals. Deleuze and Guattari
thought of man in plain and simplified terms as sheep, ``repressed''
by their urge to fit in, not empowered to do anything that ``outs''
them, as it were, as individuals. This in itself could be said to be
the fault of capitalism. Deleuze and Guattari's other main point was
that capitalism---the rising up of society's elite, through the free
markets and the substantial profits of the businesses---made society
repress its (unconscious) members without them even realising. Hence,
this was termed ``schizophrenia'', the definition of which in modern
mental health terms is that of ``changes of behaviour; muddled
thoughts; unusual beliefs not based on reality'' (NHS Choices,
2014). In a sense, the term still carries the same weight in Deleuze
and Guattari's meaning these days: capitalism, and the associated
consumerism, can blind people. It can distract them from doing good,
make them self-centered, yet also ``repress'' them so that they feel
nothing more than corporate drones, doing a job until they die, and
they have nothing left to live for. There are ways to escape from this
monotony, but doing so requires self-awareness and also an awareness
of politics, as the concept of the self and the other can be thought
of as not just philosophical, but having reaches into the
socio-political space, too.\\

Politics has a role to play in the concept of ``otherness'' as well as,
definitely, in capitalism. As we know from current society, rich
people from rich families do well---especially with a Conservative
government in power, even if they do purport to support
``working-class`` families, slyly turning this into ``hard-working''
and slamming the young, the poor and those who are disabled.
Politically, capitalism is a \textit{laissez-faire}, or free, approach
to society: free markets, free trade, businesses can just start, there
are very few controls in place. This is in order to boost the particular
society's economy and its citizens' supposed wellbeing by boosting
their artifical levels of desire. In history, desire has been seen as
a bad thing, but most political decisions are made based on whether or
not people like each other, and love is a facet of desire. Of course,
love is not objective, but that facet of desire brings us back to the
concept of ``otherness'' as decisions are not made selfishly, but in
an almost utilitarian way: decisions are made such that they enable
the provision of ``the greatest good for the greatest number'' (Mill,
J.\ S.\, 1861). Politics and otherness also have strong links. These
are apparent in so-called identity politics, as alterity is primarily
about identity: people have been influenced by politics, by the people
they see around them, by the lack of diversity of their politicians
and by the same things the politicians, or highly powered
businesspeople, do multiple times and actively go against the wishes
of their electorate purely to gain more power or, in the latter case,
appease their shareholders. This again feeds the feeling of repression
in the runt of society, because they will never rise to those heights,
it is all about heritage, and capitalism is not helping them because,
as Deleuze and Guattari said, they are just feeding a ``machine'' by
buying into the (unavoidable) construct that is their view on life
imposed on them by what society has become.\\

Now there has been some analysis of capitalism, otherness and its
effects, for better or for worse, it is important to consider
alternatives to capitalism that can foster a fairer society, as
Deleuze and Guattari were alluding to in their anti-Marxist writings.
Of course, as in China (but to a lesser extent these days as the
country gets richer), there is communism. This has its upsides, at
least theoretically: in a communist society, everyone is equal. There
are no markets, as such, because the people buy the same things: the
same hairbrushes, the same brands of food, and everything is under
state control. However, now we have only scratched at the surface,
because for there to only be one brand of food (indeed, there would
only ever be one if everyone in a country had to have the same!) or
one type of hairbrush on offer, to continue the use of banal
examples, someone has to decide which types of products will be
State-sanctioned (allowed), unless the government or regulating bodies
decide to go the very long-winded route of surveying the entire
population and averaging all the responses. That would be
\textit{democratic communism}, in its way, and that is fine for
capitalism because the government is trusted to take these steps,
perform these measures, and the government has been elected. This is
rarely the case though, as with China: most communist regimes are also
dictatorships. Whether the government be democratically elected or not
is somewhat irrelevant to the point, though, because someone has to be
at the top to impose this regime on the population, ``their''
country's society. Therefore, communism, also theoretically breaking
down barriers between rich and poor and decreasing the corruption of
big businesses that repress and control society, as previously
evoked, as Deleuze and Guattari like, is no better because there is
always an imposition from above: the person or set of people who
govern what is on sale and who can operate as a business are still
fundamentally in control of that, and they are not going to relinquish
control because if they did that, society would be tumultuous and
anarchic.\\

In conclusion, capitalism and the concept of the self and the other
are related: from selfishness to benevolence, giving jobs to people,
the so-called schizophrenic mindset of those in the grip of a society
that is mad about money and mad about personal fulfilment while
thinking that money is fulfilment in itself. When Deleuze and Guattari
wrote about the anti-Oedipus complex, refuting Freud's
psychoanalytical theory of people which was already waning in
popularity, it is sure that they did not expect their work to still be
in use today to such a relevant extent: in fact, it has got worse, as
evidenced by the above. Overall, Deleuze and Guattari seemed to get
most of their analysis right in terms of how society would pan out,
how it had started to pan out back in their day, and the pitfalls of
capitalism.

\cite{*}
\bibliographystyle{plainnat}
\bibliography{second_assignment}

\end{document}
