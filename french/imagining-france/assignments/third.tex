\documentclass[12pt]{article}

\usepackage{fullpage}
\usepackage[round]{natbib}
\usepackage{url}
\usepackage{titling}
\usepackage[official]{eurosym}
\usepackage[utf8]{inputenc}
\usepackage[T1]{fontenc}

\setlength{\droptitle}{-10em}
\linespread{1.3}

\begin{document}

\title{``There is something morally cowardly about a film that seems to
         whisper about an atrocity rather than shouting about it.''
       Discuss with reference to one or both of the films studied on
       this module.}
\author{Isabell Long (12945093)}
\maketitle

Cowardly is an interesting way for a film to be thought of. Films are,
by their nature, meant to be informative or funny, and have a central
plot line which is echoed very strongly, in order for people to want
to go and see them.\\

The film \textit{Muriel, or the Time of Return} (hereafter abbreviated
to \textit{Muriel}), released in 1963 and directed by Alain Resnais,
is one such apparently cowardly film. One of its overarching themes is
that of dishonesty, and another is guilt. The second film,
\textit{Hidden}, is similar although produced much later in 2005 by
the director Haneke who was previously famous for his portrayal of
issues that are ``provocative via polemical argument'' (from `Haneke,
Film as Catharsis' via \citeauthor{on-haneke} (\citeyear{on-haneke},
p.\ 155)). Both of them never really get to the point of discussing
their softly referenced theme of the Algerian War apart from in a very
roundabout way, or indeed the reasons or conclusions for their main
plotlines, although said soft references increase in their intensity
upon having viewed the film multiple times and later with background
knowledge of the atrocities.  That said, many of the viewers'
questions raised by the characters' actions in the film related to
their main plotlines never get answered either.  The two films are
definitely ones that generate more questions than they answer, even
with the help of cinematic and literary devices like changing of
viewing angles, flashbacks, different viewing devices such as the
kaleidescope used by Bernard when conversing with Françoise in
\textit{Muriel}, voiceovers such as the one at the very start of
\textit{Hidden} which is said to bring a sense of reality to the
potentially surreal beginning of just a static shot of a house which
stays static for an eerily long time \citep{cache-meanings}.\\

\textit{Muriel} is a film that starts off slowly, with many supposedly
intertwined plot lines. A family of which the mother --- Hélène ---
meets an old flame and invites them for dinner, which kickstarts the
web of lies.  To give an example, the old flame, Alphonse, brings
along his ``niece'' Françoise to this encounter, who talks in depth to
Bernard --- the war-torn stepson of Hélène who has recently returned
from fighting in Algeria, and as dramatic irony would have it it
transpires that the young woman is not a niece but a lover. Alphonse
lies prolifically, telling Bernard that he has been in Algeria too and
that he was ``surprised they never met''. This is an example of a
quaint social interaction around a dinner table --- something so
common in France --- and people trying to please each other without
seeming overly superior.\\

\textit{Hidden} is itself full of dinner table scenes, but they are
not given any attention --- the camera briskly moves to the next scene
which is often completely unrelated to the last. Nothing is explicitly
violent at first, but confusing --- again reinforcing the ``cowardly''
viewpoints. The pinnacle of the film is widely regarded \citep[p.\
22]{cache-book} to be when Georges, the lead character, has a
nightmarish flashback to his childhood during which he incited his
friend Majid to kill a chicken. The viewer is shown this memory in
full colour and gore, which is most definitely not shying away from
the event. It could be said that it is intensified by the final shot
of the nightmare before he wakes up drenched in sweat being Majid
wielding the same axe covered in chicken blood over his head,
threateningly. It is almost as if Georges anticipates what is coming
next with the videotapes as they get more scary and more frequent,
accompanied by crayon drawings, and the police do not want to help him
identify their source or reason. This has parallels with the police
not wanting to help during the massacre in Paris in 1961, because they
did not see evidence that anything bad would have happened at the time
it was required that they decide whether to intervene.\\

Like war in real life, the scenes in both of these films are very
jerky and keep the viewer guessing as to what the next revelation or
the next hurt will be. \textit{Muriel} provides different explanations
every way you turn --- in fact almost in every shot --- and this is
reminiscent of war itself being more often than not a web of lies and
propaganda for the richer, more powerful, better armed country of the
group. Indeed, the beheading of the chicken scene in \textit{Hidden}
also can be said to reinforce the thoughts of bloody acts of war and
war's brutality, as well as the innocence of some children. In that,
it is obvious to the viewer that it is not just adults that are
subjected to atrocities, or even commit them, but children too, and it
children remember more than we think they do, hence Georges still
being haunted in nightmares in adulthood by gruesome things that took
place in his relative infance.\\

On the subject of morals, as per the title, it is important to think
about what exactly morality is. As per the Oxford English Dictionary
in \citeyear{oed-morality}, ``morality'' is defined as ``[p]rinciples
concerning the distinction between right and wrong or good and bad
behaviour''. Moral cowardice is the principle of, for example, a
person not wanting to face up to their wrongdoings in life and
society, doesn't want to take responsibility for their actions, or to
do what is right when it is the hard thing (potentially lacking in
social norms) to do. When analysing these films, the critic or even
the director could consider whether it is better to leave the viewers
to think for themselves and draw their own conclusions, or spell
everything out to them and emphasise the exact points they are making.
If they did that, however, viewers would not be very happy as it has
been shown by studies on viewing habits and researchers such as
\citeauthor{film-cognition} (\citeyear{film-cognition}) that viewers
do not like to be kept guessing --- they watch TV to relax or have
fun.\\

The point of view of the films being morally cowardly could be said to
be unjust, in stark contrast to earlier evidence and ensuing thoughts.
As expressed in Matthew Croombs' essay \textit{Algeria Deferred: The
Logic of Trauma in Muriel and Caché} (2014), according to the famous
psychoanalyst and psychologist Sigmund Freud, trauma is not
immediately obvious at the time of the event, but at the time that a
later discovered traumatic event occurs. An event requires a second
stimulus --- a memory or another, similar event --- to be recognised
as traumatic and can then be hellish for the person experiencing the
psychological or physical effects. Therefore, it would have been hard
for either Resnais or Haneke to have made the films directly about the
Algerian War. For one, many French people did not realise the
atrocities were happening because of the propaganda via broadcasting
news mechanisms, and events recalled in film would have been too
shocking for many viewers to have ever gone to see the films in the
first place, even forty years after when it had gone from (or not
reached) the psyche of most French people. Indeed, at that time there
were still bans on books discussing the French massacres in Algeria.
These included Henri Alleg's book \textit{La Question}, or the
philosopher Satre's book \textit{Les Temps Modernes} which was banned
from sale by the French government straight after its publication,
presumably because it directly touched on these issues of the massacre
that were still required (or wished) to be covered up. This explains
why neither of the films directly touch on the issues: neither Haneke
or Resnais would have wanted their films banned, as the idea of a film
is to diffuse into the minds of the most people possible, and the
Algerian War was a very important topic that they believed should not
have been subject to censorship. Therefore, they could be said to not
be morally cowardly, but simply careful: they wanted to appeal to the
general public and have their films allowed to be shown for years to
come.\\

These films supposedly ``whispering'' are doing so very cleverly. As
the violence mounts in \textit{Muriel}, this is akin to the violence
mounting in Algeria, like the French government refusing to let
citizens know that there was anything to worry about and instead
insisting that it was ``under control''. The name of the second film
\textit{Hidden} (\textit{Caché}), is quite poignant in light of the
censorship and ``hiding'' of the atrocities in Algeria by the French
government and media. What the film is trying to hide, the viewer
never knows --- as with the Algerian War, since secrets were rife.
This is also emphasized in \textit{Muriel} with the vague mentions and
never a full revealing of the eponymous girl Muriel --- Bernard just
talks about her hauntingly, leaving the viewer to ponder what happened
to her. This draws another parallel with the War: the families of
those killed in the Algerian War (or indeed the Paris massacre when
refugees were pushed into the river) never got closure or even an
explanation because the French never admitted to wrongdoing.\\

Wrongdoing is an interesting word here. In the aforementioned scene of
\textit{Hidden} --- children beheading a chicken --- and when Georges'
mother pretends that she does not know Majid, because she does not
want to remember the realities of that friendship, what happened to
his parents (they were killed in the Paris massacre of Algerians in
1961), and the broken promise of adoption by Georges' mother that was
due to her precious son having lied to save himself a scolding for
having incited the act of violence. In this case, denial is the
response, but it comes back to bite Georges with the videotapes and
the disturbing images, flashbacks and nightmares --- who is watching
his house, and why are they --- why are they trying to terrorise him
and his family? We can also ask whether lying should have terrifying
consequences such as these hauntings and invasions of privacy.
\textit{Muriel}'s portrayal of wrongdoing, betrayal, misplaced loyalty
(Hélène with Alphonse) and secrets, such as Hélène not realising
Alphonse was married until he was asked to return home
\citep{muriel-blog}, and hence her dreams being shattered by these
overarching themes of dishonesty and disloyalty.\\\\

It can be concluded from a study of these two films and their events
that indeed, they do whisper about ``an atrocity'' and that atrocity
is indirectly the Algerian War. In both films, the main plotline is
not that of the War, but the viewer gets sucked in to watching growing
chaos as both main characters in each of the respective films try to
piece together what has happened. The main thesis of both films in
this respect is that we can never really know what happened before if
we did not directly experience it --- there may always be some machine
spinning its own version of the truth, and deciphering what is
actually the truth is taxing on the brain.

\bibliographystyle{plainnat}
\bibliography{third}

\end{document}
