\documentclass[12pt]{article}

\usepackage{fullpage}
\usepackage[numbers]{natbib}
\usepackage{url}
\usepackage{titling}

\setlength{\droptitle}{-10em}
\linespread{1.3}

\begin{document}

\title{Question D.\\
        \small{Provide an account of Broca's aphasia. Detail how this
               syndrome was first identified \& its neurological \& linguistic
               characteristics.}}
\author{Isabell Long (12945093)}
\maketitle

Broca's aphasia is a neurological condition that occurs in the frontal
lobe on the left side of the brain---the Broca's area---which can lead
to the sufferer not being able to coherently form complete sentences,
as the Broca's area is a key region of the brain for speech
production. Broca's aphasia usually presents itself after significant
brain damage, such as a stroke or head trauma. The Broca's area was
discovered by---and therefore, as was common at the time, named
after---Paul Broca, a French surgeon and neuroanatomist.\\

Paul Broca treated a patient---his first---initally known as Tan, so
named because all he could articulate was ``tan''.  Broca determined
that Tan had a ``syphilitic lesion in the left cerebral
hemisphere''~\cite{wikipedia-lesion}, and so this type of aphasia was
identified with that as a characteristic. This patient was later
identified as 51 year old Leborgne who ``had been without speech for
many years''. It was found in Broca's research on him that he could
``vary the intonation of sound, [but not] produce any recognizable
words or phrases''~\cite{tan-postmortem-imaging}. ``Tan'' is a word
these days for a dark yellowish colour, but it was undefined in that
form in the 1860s when Broca was conducting his research, according to
the~\citeauthor{oed-tan} which states that the definition of it being
a colour only came into existence in 1888. To provide further support
to his theory that the frontal lobe of the brain is involved in
language production, specifically in his eponymous area, and following
Leborgne's death, Broca concentrated on a French stroke patient Lelong.
Lelong could say only five words, and those words always in
succession: ``yes'', ``no'', ``three'', ``always'' and his mispronounced name,
``Lelo'', obviously in their French
translation (also
\citeauthor{tan-postmortem-imaging})~\cite{tan-postmortem-imaging}.
Broca received great acclaim for his findings, and advanced
neuroscience greatly, giving scientists foundations on which
to learn from autopsies of brains~\cite{broca-scientific-american}.\\

Patients with Broca's aphasia generally have only basic problems
understanding sentences, but later evidence suggests that they miss
out key articles such as ``is'', ``the'' or the past-parciple ending
``-ed'' when forming their own sentences in
response~\citep{articles-omitted-primary} and can be very slow. For
example, a sentence ``the sky is blue'' would be articulated as
``sky\ldots blue'', because the sufferer has lost knowledge of
grammatical constructs in their language. In linguistics, this is
known as ``agrammatism'': ``[l]oss of the ability to use correct
grammar''~\cite{oed-agrammatism}. Broca's aphasia affects not only
language production as agrammatic sentences, but also muscle control
such as the person's ability to move their mouth to articulate
sounds. \citeauthor{nhs-choices-brocas-aphasia} states that indirect
symptoms can include ``weakness down one side of [the patient's]
body (usually the right side)'', most likely due to lateralization of
brain function---in that the left brain---where the Broca's area
is---controls the right side of the body. This is why when people are
suspected of having a stroke, they are asked in the
`F.A.S.T'~\cite{stroke-fast-phe,stroke-fast-nhs} steps to put their arms
out in front of them: to check that they are still able to control the
movement in both sides of their body.\\

This type of aphasia can be contrasted with another---Wernicke's
aphasia, that affects the cerebral cortex area of the brain---in which
the sufferer does not understand sentences and completely says the
wrong word. To give an example, a ``fork'' could be said to be a
``geeble''~\cite{fork-geeble}, which is not a real word and most
certainly does not bear any relation to the sound of the word
``fork''.\\

Diagnosis of Broca's aphasia compared to Wernicke's aphasia is usually
completed by a neurologist following the sufferer's injury. Almost
instantaneous recovery is possible but reasonably rare: language
abilities (whether they be speaking, reading or writing) can return
within weeks~\cite{aphasia-treatment}, however more often than not it
takes months or years of following a careful treatment plan.
Apparently, 40\% of adults who develop this type of aphasia recover at
a steady pace~\cite{gradual-recovery}. Treatment of this impairment
usually involves speech and language therapy conducted at a hospital
by a trained specialist, but each case is different and patients need
unique, tailored treatment. Techniques can include asking the sufferer
to name objects such as horses or dogs, or computer work to readjust
the patient to reading words in the case that their speech has not
been too impaired. It is important too to involve other family
members, if the patient has any, so that he, she or indeed the family
members, do not feel isolated. The patient's confidence must then be
increased by practicing normal day-to-day tasks such as greeting
people and making phone calls. If the person has never been very
talkative but, for example, likes singing, there is
evidence~\cite{singing-recovery} to suggest that singing words in
sentences, rather than speaking them, increases the treatment success
rate because it's a different method of expression. In the study by
\citeauthor{singing-recovery}, normal sentences such as ``the sky is
blue'' were sung instead of spoken, using a technique called melodic
intonation therapy, the right hand motor controls were engaged with
hand-clapping to ready the left of the brain to articulate the words,
and with this fewer errors or hesitations occurred compared to the
control group. This leads us to remember that not only is singing good
for us, according to the \citeauthor{singing-happy}~\cite{singing-happy},
but different intonations and an appearance of fun may play a key role
in comforting a sufferer. It may possibly lead him or her to not think of
the condition as so disabling and emotionally debilitating, as they
may not have to put so much pressure on themselves to make
sounds~\cite{non-fluent-singing}.\\

Overall, Paul Broca was most certainly not the first to come up with
theories of why language problems such as these occur, but his
research was the most thoroughly
documented~\cite{tan-postmortem-imaging} (compared to Marc Dax, as
found in \cite{dax-brocas} via \citeauthor{tan-postmortem-imaging}, or
many of the other paper
authors in~\cite{mit-brocas}). So, because science must be
replicable, his findings stuck. The 1960s were founding years for this
part of neuroscience, but it doesn't stop there: 2014
research~\cite{2014-brocas} continues to uncover results which will
mean more questions before answers.

\bibliographystyle{plainnat}
\bibliography{question_d}

\end{document}
