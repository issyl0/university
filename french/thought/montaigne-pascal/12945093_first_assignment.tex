\documentclass[12pt,a4paper]{article}

\usepackage[utf8]{inputenc}
\usepackage[T1]{fontenc}
\usepackage[numbers]{natbib}
\usepackage{titling}
\usepackage{fullpage}
\usepackage{url}
\usepackage{breakurl}
\usepackage[breaklinks]{hyperref}

\setlength{\droptitle}{-10em}

\begin{document}

\title{Why does Montaigne emphasise pleasure and happiness when he
talks about education?}
\author{Isabell Long (12945093)}
\maketitle

Montaigne's best and most popular works are his set of
\textit{Essais}. Written in the 1950s, they remain influential today.
In the first volume, chapter twenty-six, ``\textit{De l'Institution
des Enfants}'' (in English ``\textit{On the Education of Children}''),
he talks of education, in particular of schools, and how education
both helps and hinders the instilling of behaviours, wisdom and
knowledge into our children.\\

Happiness is of course a very crucial part of education and the
process of educating, because without happiness and some desire to
repeat the learning experience, children, or anyone, will not learn.
The same applies the other way around: education leads to happiness
because of behaviours learnt and knowledge gained
(\citeauthor{happiness-education}, \citeyear{happiness-education}).\\

Montaigne's foray into writing about education stems from his own
experience, as do most writings, especially those of philosophers. He
states that he ``never seriously settled [him]self into the reading of
any book [\ldots] little or nothing stays with [him]''. His
introspective wisdom as a source is a common pattern even
recently~(\citeauthor{recent-relevance}, \citeyear{recent-relevance}),
showing that his literature and his experiences, although definitely
not on the reading lists of most young people due to their at first
seemingly impenetrable vocabulary, are relevant and he is to be
respected, even though he did not know this at the time. His goal in
writing the \textit{Essais}, however, was initially set out to
``describe man with utter frankness and honesty'', and that stands the
test of time---man (the population) is not going to change overnight,
or even in centuries---we will still have the same problems, albeit
more advanced in their solutions and differently approached.\\

Education enables us to think knowledgeably and insightfully not only
about our own lives and experiences, but also about others' lives and
experiences should we choose to, or even unconsciously when evaluating
our own. Increased awareness and introspection can lead to us
realising potentially awful things about people (although usually
oneself), in the interim period between discovering said things and
dealing with the traumas that they may bring up. Montaigne gave
evidence for this in a way in the same essay, by saying that
``[w]onderful brilliance may be gained for human judgment by getting
to know men'', where ``men'' here, like before, can also refer to your
own being --- «~\textsl{le soi}~». Just talking or writing about one's
own experience or thoughts can be a key to finding peace or
happiness~(\citeauthor{baikie2005emotional},
\citeyear{baikie2005emotional}), the subjective or even imaginary,
socially constructed concept that it may be to give us some sense of
purpose as we move through life.\\

Montaigne's stance on Latin was interesting and again quite
revolutionary for his time. In his era, Latin was the language of
legalese and still to a large extent education, and he was forced to
learn it by his learned, monied parents: ``it was an inviolable rule,
that neither himself, nor my mother, nor valet, nor chambermaid,
should speak anything in my company, but [\ldots] Latin words.'' He
includes copious quotes in Latin throughout all his essays, perhaps to
make himself seem traditionally educated and have his views more
universally respected. His choice of quotes are almost all from one of
Montaigne's greatest influences:
Plutarch~(\citeauthor{friedrich1991montaigne},
\citeyear{friedrich1991montaigne}, page 71), who was revolutionary in
his own way with his writing many years earlier---but not
revolutionary enough for Montaigne who could always try harder.
Montaigne attacks the State over its methods of teaching everything.
Montaigne viewed teaching methods of the time to be too prescriptive,
not leaving enough time to students for recreation, original thought,
and for non-academic subjects and past-times such as reading fables or
poetry, for which he declared at the start that he had ``a special
affection'' and which can also enrich the learning experience while
also being an enjoyable hobby and escape from everyday problems, even
at that young an age. This was partly down to his parents, as he
states just after the passage on Latin that he was taught Greek from
his father ``by way of sport, tossing our declensions to and fro.'' He
goes on to give some context to his views on teaching methods, saying
that his father ``advised to make me relish science and duty by an
unforced will, and of my own voluntary motion, and to educate my soul
in all liberty and delight'', where ``liberty'' is a key adjective
here as it goes to show that children should not be constrained by an
adult's way of thinking.\\

Relatedly, he quotes Plato in this particular essay on education:
``[t]he authority of those who teach is very often a hindrance to
those who wish to learn'' (page 55). There is always an underlying
fear in any student of teachers getting angry at a wrong answer, and
pupils not wanting to respond for fear of being labelled as
unintelligent, or ridiculed. These things are most definitely not
happiness. In using Plato's quote, he refers here to the methods of
scholar Socrates---of whom Plato was a student---who always let his
students speak first before speaking with any preconceived judgements
or prejudices himself, either directly or indirectly, so as to not
influence them, thereby provoking entirely original thought. This
gives power to the student, and more confidence, as can be seen in a
research paper by \citeauthor{holden2002inquiry} in 2002. Both
Socrates and Montaigne admit readily that they do not consider
themselves knowledgeable, simply wise from thinking. Socrates famously
admitted ``I know that I know nothing'', and Montaigne followed that
with similar: ``[n]othing is so firmly believed as that which we least
know'' (from Book I, chapter 32). Also, as if to reinforce the point,
he asks a rhetorical question in another of his essays---chapter
12---``[w]hat do I know?'', thus reinforcing his overarching sceptic
ideologies. This self-confidence---alternatively viewed as
self-deprecation---strange as it may seem, leads the person to not
have an inferiority complex, because they know that they are ignorant
yet wise in realising that in reality they have as much reason as
anyone else to feel superior---that is, none---because we all fight
our own battles and ``know'' different things, however flimsy that
word may be now. His reference to olden days, even then, with Plato
and Socrates, gives weight to his arguments because they have been
proposed or supported previously by respected thinkers. It is also
important not to think we know too much, or boast about what we
believe to know, as ``[t]he plague of man is boasting of his
knowledge'', and this is an important lesson about humility for
everyone, not just those in education, because boasting can cause
superficial happiness (\citeauthor{khattak2014morality},
\citeyear{khattak2014morality}) after the time we have outsmarted or
put down our friends or enemies.\\\\

He again infers happiness and moral enrichment as the purpose of
education rather than blindly memorising facts or numbers: ``an
accomplished man [\ldots] [is] better than a scholar'' (page 54). The
ideal pupil should ``taste things, select them, and distinguish them
by [their] own powers of perception'' (also page 54) and overall that
``too much knowledge could prove a burden''~(Stanford Encyclopedia of
Philosophy, 2014\nocite{montaigne-seop}), where knowledge in this
case, as mentioned in the paragraph above, comes from books or tutors,
not the inquisitive minds of the children themselves. This then calls
on the child's natural abilities and senses, not just the metaphorical
``regurgitation'' (\textit{« regorger »}) of facts---a good digestive
metaphor for lapping up ideas without thought as to their origins,
like we do with food when we are very hungry, or water when we are
parched. Overall this viewpoint justifies his sustained emphasis on
education through philosophy and thinking for oneself throughout every
one of his essays on education. Indeed, it is still relevant in
today's French society, especially in later school years with
philosophy being an important and highly weighted subject in the
\textit{Baccalaur\'eat}.\\

Montaigne criticises not only all of the above, but class sizes and
group-based learning (\citeauthor{fiche-bac}). These criticisms are
still perfectly valid today, with class sizes averaging 26 pupils at
all stages of education in a 2011 report by the Department for
Education\nocite{dfe-class-size} and strictly limited to 30 pupils for
ages five to seven ``to help raise standards''
(\citeauthor{govuk-hse-school}, \citeyear{govuk-hse-school}). A
smaller class size is presumed to enable more focused, one-to-one
teaching, which is vital in the early years of education, and apparent
in \citeauthor{krassel2014class}~(\citeyear{krassel2014class}) and
anecdotal evidence from students progressing from primary to secondary
school in Hong Kong (\citeauthor{hk-anecdote-class-size} in the
\textit{South China Morning Post}, \citeyear{hk-anecdote-class-size}).
This comes back to happiness in that if a child feels valued in his or
her lessons, they are more likely to feel
happy~(\citeauthor{barragan2008bienestar},
\citeyear{barragan2008bienestar}). Montaigne's recommendations and the
aforementioned associated evidence for them have gone a long way
towards proving that he was right all those years ago.\\

On a more personal level, not explicitly about education but leading
on from his long treatise about it and in the same essay---``On the
Education of Children''---, Montaigne states that ``[m]ixing with the
World has a marvellously clarifying effect on a man's judgement,''
emphasising patience, adventure and thinking outside of the box as it
were. Yet, in the previous few pages he had lamented man's selfishness
with the attack: ``[i]n this school of human intercourse there is one
vice that I have often noted; instead of paying attention to others,
we make it our whole business to call attention to ourselves.'' It is
possible that he meant that in order to feel fulfilled, we need to
think of others before ourselves, like the maxim ``do unto others as
you would have them do unto you''---treat others how you would wish to
be treated by them. This would have been fairly common as Montaigne
was a religious man, a Roman Catholic, yet a sceptic, but still
believed in the right way to do things, hence his copious amounts of
writing about life in his \textit{Essays}. ``Let an honest curiosity
be instilled in [the child], so that [they] may inquire into
everything'', he writes on page 61, again emphasising individuality,
patience and self-providing happiness through wisdom. ``Honest'' is an
important adjective here because without honesty, no-one can be truly
happy, althogh some would argue that sometimes lying is better for the
other person~(\citeauthor{lying-better}, \citeyear{lying-better}).
Eventually this eats away at people, however, leading to unhappiness
and a sense of regret, leading to more unhappiness later in
life~(\citeauthor{lying-unhappy}, \citeyear{lying-unhappy}) hence the
common teaching that ``honesty is always the best policy''. ``Honest
curiosity'' is interesting too, combining the two central words to
this quote, because it implies that some curiosities or interests are
potentially adult-led, adult-imposed, and things that do not entirely
interest the child, which could either eventually brew and have them
eventually like them and be happy about this newfound knowledge or
skill, or become even more unhappy because they are being forced to
study, read or play something they hate.\\

Overall, happiness was a very large part of Montaigne's teachings,
analyses and stance on life. His opinions and suggestions throughout
all of his Essays, have been influential in planning the best modern
education. He was a sceptic, so had to find happiness anywhere he
could to save himself from despair at the rest of the world, and that
came through freedom of thought (a cornerstone of scepticism itself)
and liberty to do what he wished. That could be seen as at odds with
education, because the classroom is set up with constraints, but those
are only physical, not mental, and pupils should be encouraged to
challenge them in the pursuit of happiness and pleasure.

\bibliographystyle{plainnat}
\bibliography{first_assignment}

\end{document}
