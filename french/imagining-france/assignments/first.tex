\documentclass[12pt,a4paper]{article}

\usepackage[utf8]{inputenc}
\usepackage[T1]{fontenc}
\usepackage{titling}

\setlength{\droptitle}{-10em}

\begin{document}

\title{Commentary on Passage A.}
\author{Isabell Long (12945093)}
\maketitle

Françoise de Graffigny's text \textit{Letters of a Peruvian Woman} was written in the 1700s, about a Peruvian who is captured and ends up in France. She never gives up hope of finding her lover and husband-to-be, Aza, and writes letters to him detailing her torment and new discoveries. The extract for analysis is letter twelve near the beginning of the book, when the protagonist Zilia is dressed as French women dress, to make her seem less different and fit in with societal norms.\\

It can be seen even from the first paragraph that Zilia does not feel quite at home in the extravagant home of Déterville. On the first line, she describes the costume as ``most beautiful'', the superlative ``most'' inferring that she had never seen anything quite so beautiful in Peru --- or, indeed, that it was a different kind of beautiful. She then goes on to mention how it is a costume worn in ``his country'', and therefore that she does not yet or at all consider France her home. She discovers various new devices, in this case a mirror which she calls an ``ingenious'' machine. This adjective shows that she has no idea how the mirror works, but she finds it incredibly useful to be able to see herself and seems pleased with how the dress looks once the servant has arranged it correctly. She was probably capable of doing that herself, even though it was a new piece of clothing to her and worlds removed from what she is accustomed to, but French society and its behaviour towards upper class women lets them be handled and doted upon by servants. Zilia further notes the ``tiresome looks'' she was used to getting from the French which made her feel looked down upon. The adjective ``tiresome'' could both refer to her feelings, or the French people's feelings about her differences. There was still quite a stigma around Peru when this book was written, with some people not even regarding it as a proper country, as noted in the historical introduction to the novel.\\

Throughout her series of letters, Zilia is very direct in what she says, unlike French women who are very measured and think before they speak. She is also clearly used to men saying what is plaguing them and about their love interests. In the next few paragraphs, she puzzles over Déterville's embarassment at wanting to see the dress he gave her --- this is a perfectly natural thing to do and she herself is ``alarmed by his state of mind'', in the end deeming his behaviour ``incomprehensible''. The reader can infer that this incomprehensibility of behaviour is a cultural difference, due to the aforementioned directness of Peruvian people. That said, unlike earlier in the passage where she laments other men staring at her, she seems pleased that Déterville should do this and seems to regret him stopping himself from embracing her, as not doing so is clearly causing him distress of which she is the subject. This restraint on his part is because he knows and sadly accepts that her love is for Aza only, and yet Zilia remains confused --- she ``fe[lt] uncomfortable without knowing why''. This is potentially slightly confused, however, as Zilia infers that the ``phrases which he enjoys to have [her] repeat'' are pleasantly emotion-based, as Déterville's ``cheeks redden''. She only ``repeats'' them because her level of French at this stage of the novel is still quite basic, as seen from the flowery, very correct language she uses to write these letters. This occurrence of her only repeating the words he has told her in the past could lead Déterville to feel worse about his feelings, as he knows that she doesn't fully understand and is not eloquent enough in French to really connect with what she is saying. In this way, she could be perceived like a baby who can mimic sounds but who does not necessarily understand meaning without a great amount of visible context and repetition of phrases.\\

In conclusion, this extract shows Zilia's weaknesses and her desire to do right by people --- even her captors. She dotes on Déterville and loves him like a brother, this causing both him and her signs of distress as she cannot do more to appease him. Her writing in the passage touches on some of the troubles of women in French society, but also begins to appreciate their standards and class of living --- she does not disregard the dress, for example, but wears it with pride. The ending of the passage where she deems behaviours ``incomprehensible'' sums up her views on France and the French language as a whole, not just Déterville's melancholia --- she feels lost without great means of native communication, hence her seeking solace in writing letters to Aza, her lover.

\end{document}
