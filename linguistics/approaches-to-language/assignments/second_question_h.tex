\documentclass[12pt]{article}

\usepackage{fullpage}
\usepackage[round]{natbib}
\usepackage{url}
\usepackage{titling}

\setlength{\droptitle}{-10em}
\linespread{1.3}

\begin{document}

\title{\small{Detail the types and process of language planning.
              Provide an account of one case study to illustrate how
              language planning may take place and influence language
              policy and practice.}}
\author{Isabell Long (12945093)}
\maketitle

Language planning is a practice considered to be about changing
language --- for better or for worse --- either to add new words for
certain types of foods or behaviours for example, or to develop new
languages, to name a few of many possible activities. It can also be
used strategically by a country's leaders to select its official
language or multiple official languages, or indeed campaigns from the
general population. The concept of language planning and this term for
it was coined by an American linguist Einar Haugen in the late 1950s.
Over the years since 1960, linguists have taken a great deal of
interest in language planning, with some deeming it detrimental to the
natural evolution of language if, for example, new grammars or words
are imposed by those part of high up groups or authorities \citep[p.\
67]{language-planning-detrimental}. Language groups such as the French
\textit{Acad\'{e}mie Fran\c{c}aise} are doing very well, however ---
they control to some extent what goes into the French dictionaries and
draw up grammatical rules.\\

There are many types of language planning and many processes that can
be used to perform it. There exist four types of language planning:
corpus planning, status planning, prestige planning and acquisition
planning, which are all elaborated on below.\\

Corpus planning deals with the internal structure of language, and its
central purpose is creating and establishing linguistic norms such as
grammar and spoken words \citep[p.\ 372]{sociolinguistics-mesthrie}.
This becomes especially useful when there could be different languages
taught for religious or educational purposes such as in schools or in
places of worship, for example classical Hebrew for religious
scriptures and modern Hebrew used for education and in day-to-day
life. This is partly because classical Hebrew in this case is missing
science and technology-related vocabulary which are becoming more and
more important in this day and age.\\

Language planning of the status variety is about people making efforts
to change a language's use within the realm of people who speak or
write the language. This could be done by varying the language people
learn in --- from Dutch to English for example, if English is deemed
the language of the world and if it is deemed important that everyone
be immersed in it. It has been shown that immersion has a key part to
play in language learning, so instructing pupils in the target
language is, some say, a good idea \citep{immersion-education}.\\

Prestige planning, the penultimate type of language planning, is often
used by governments to ``influence how language is perceived'' and to
choose or determine by research what level of respect is given to the
particular language \citep[p.\ 296]{soas-language-planning}. Prestige
planning can be used with --- or against --- both speakers of the
language or non-speakers of the language.\\

Acquisition planning is perhaps more self-explanatory. This final type
is to do with how inhabitants of the country end up learning the
language. Usually brought about via schools, with lessons being taught
in the new language --- partially as with status planning ---,
``language-in-education planning'', as it is widely known thanks to
\citeauthor{language-kaplan} in \citeyear{language-kaplan}, concerns
itself with books being published, recommending or blankly changing
teaching methods, or deciding how much money should be spent on
introducing this new language \citep{language-kaplan}. That said, it
is interesting to note that the framework for this type of planning is
the only one to heavily and directly mention money. It could be argued
whether money should come into influence or question when we are
evaluating or changing something so fundamental to society as language
use and adoption. Here, it is worth noting that poor countries still
manage to harness the education system generally, with outside
support, and don't need much money with which to do it, therefore it
might be a moot point that language planning and therefore teaching in
a language requires sizeable financial investment
\citep{developing-countries-primary-education}.\\

The processes of language planning lead on from the aforementioned
types they are related to. A framework was developed in order to
describe these processes of language planning, and it contains four
stages: selection, codification, implementation and elaboration
\citep[p.\ 375]{sociolinguistics-mesthrie}.\\

In the first stage --- selecting language --- one specifically chooses
a language or a dialect to enable the performing of certain functions
in society or group of people. Unfortunately, again according to
Mesthrie et al.\ (2009, p.\ 375), it is ``usually the most prestigious
dialect or language [that] is chosen [\ldots] for example Parisien
French''.  This is unfortunate because it may make some minority
dialects, for example ``la langue d'Oc'' or ``Oil'' feel less valued
or not at all recognised, and impact their ability to be able to be
taught in schools in those areas of a country (again in this case,
France).\\

The second stage as mentioned above --- codification --- also has
sub-stages related to the development of written communication,
grammatical rules and vocabulary specification which all help to
fulfil its purpose of ``creating a linguistic norm''\citep[p.\
375]{sociolinguistics-mesthrie}. These are named graphisation,
grammatication and lexicalisation respectively --- ``lexicon'' of
course meaning ``[t]he vocabulary of a person, language, or branch of
knowledge'' \citep{oed-lexicon}. The next stage --- implementation ---
could be argued to be fairly self-explanatory: it involves
implementing the new language in ways such as producing literature and
introducing it into the education system as a formal, taught language
or a language in which lessons are taught \citep[p.\
378]{sociolinguistics-mesthrie}.\\

The fourth and final stage is sometimes referred to not as elaboration
but as modernisation \citep[p.\ 379]{sociolinguistics-mesthrie} and
involves --- as the second name suggests --- modernising languages to
``meet the needs of modern life and technology'' (again
\citeauthor{sociolinguistics-mesthrie},
\citeyear{sociolinguistics-mesthrie}, p.\ 379), such as new words like
``selfie'' or ``computer'' in the technological age. This or any
creation of new words is known as ``neologism'' \citep{oed-neologism},
and is extremely important for the continued development of even a
mature language like English, and therefore especially so for a young
or minority language like --- to take an example of a case study from
\citeauthor{sociolinguistics-mesthrie}
(\citeyear{sociolinguistics-mesthrie}) --- Kohi, a South African
language.\\

A case study is useful to analyse when making decisions about whether
or not to use language planning. One such case study was famously
conducted by Ivar Ansen and Knud Knudsen (reproduced in
\citeauthor{sociolinguistics-mesthrie},
\citeyear{sociolinguistics-mesthrie} from
\citeauthor{haugen-planning}, \citeyear{haugen-planning}). It was
centered around Norway and the adoption of a new language ---
Norweigan --- and the cancelling of Danish as the de facto ``language
of administration, politics and education`` when Norway gained its
independence. Their research allowed them to separate so-called
country language and state language into two, each creating one of
``Landsm\"{a}l'' and ``Riksm\"{a}l'' respectively. Eventually in the
1880s the Norweigan government recognised them both and they were the
official languages of Norway, with their only real differences being
with regard to their grammars and sentence structures. This got
``awkward and impractical'' and language committees formed from the
government sought to unify the two strands of language. This caused an
even greater divide, however, between the lower classes and the upper
classes who thought that their cherished languages curated over the
years would get lost or infected with colloquialisms. These divides
are a perfect example of why language planning can end up not being
such a good idea, such a perfect solution for all, and this is where
language planning --- also known as language policy where governments
are involved in setting rules, frameworks or policies to do with the
adoption of language --- becomes contentious.\\

To combat this discontent and perceived inequality, it has been
brought into force by organisations like UNESCO as early as 1990 that
people have rights to language, so much so that a Universal
Declaration on Linguistic Rights was signed in a similar fashion to
the Universal Declaration of Human Rights \citep[p.\
390]{sociolinguistics-mesthrie}. The United States was the first to
codify a person's linguistic rights in 1991. These include being able
to communicate ``freely'' in any language they desire ``in public or
in private'' and, especially applicable to minority languages (such as
Navajo, of Native America), ``to be able to maintain their native
language and teach it to their offspring''. This is very important so
that heritage is not lost and historical records are preserved. If
children learn minority languages as a matter of course, even not
formally but passed down by their parents or grandparents, when their
elders die they could possibly translate things in history, becoming
an invaluable economic resource and allowing us to discover details
about our planet --- or indeed others --- that might have remained
buried by the sometimes overzealous actions of language planners
always wishing to modernize by destroying or devaluing systematically
a country or region's heritage.\\

The examination of the case study above illustrates why the process of
language planning is referred to by \citeauthor{language-planning-detrimental}
(\citeyear{language-planning-detrimental}) as something that ``should not be
done'', as explored in the introduction. However, it has many
advantages, most of which have been touched on. Importantly, it is
good for modernisation of countries' languages, which is very much
needed in this day and age if citizens of any countries are going to
be able to understand the world, and yet also if existing in some
cases minority languages are to be preserved in often rural or
societally excluded communities.\\

\bibliographystyle{plainnat}
\bibliography{second_question_h}

\end{document}

