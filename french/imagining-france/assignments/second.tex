\documentclass[12pt]{article}

\usepackage{fullpage}
\usepackage[numbers]{natbib}
\usepackage{url}
\usepackage{titling}
\usepackage[official]{eurosym}
\usepackage[utf8]{inputenc}
\usepackage[T1]{fontenc}

\setlength{\droptitle}{-10em}
\linespread{1.3}

\begin{document}

\title{What was the importance of the French Republican school system for the reinforcement of French national values in the period 1870--1914?}
\author{Isabell Long (12945093)}
\maketitle

The French have always placed a lot of importance on their school
system, particularly their primary school system, faming it for its
secularism or \guillemotleft~\textit{laïcité}~\guillemotright~--- the
absence of religion in state schools, therefore the separation of
Church and State, that was brought in by Jules Ferry --- and its
egalitarian principle of bringing every student up to the same
standard of education regardless of economic status or a child’s
upbringing, mostly via very few educational concessions. This
consistency was all the more important in the 1800s and 1900s, as this
was of course the time leading into the First World War when there
existed lots of political and social unrest, and the time of the
founding yet testing Republican principle of equality for all
citizens.\\

French national values have always underpinned anything France has
done. Those national values, as defined by Fénelon and
constitutionalised in 1958, are \guillemotleft~\textit{liberté,
egalité, fraternité}~\guillemotright~--- liberty, equality and
brotherhood\cite{national-values}. These are the principles that
everyone is free to do as they wish as long as it is in accordance
with the laws of the State; that everyone is equal; and that everyone
should look after everyone else to preserve goodwill. Therefore, in
France, the popular class system does not exist --- rich and poor are
the only social states and discrimination should not occur based on
so-called lower\slash upper class living or educational standards.\\

France's egalitarian principles are not to say, however, that France
is equal in every way about every thing it or its make up of French
people do. For example, numbers of students progressing to the most
prestigious universities --- or \guillemotleft~\textit{grandes
écoles}~\guillemotright~ --- such as the Saubonne in Paris is
undoubtedly and evidently higher from feeder colleges such as those
located in prestigious parts of Paris rather than in the suburbs ---
\guillemotleft~\textit{banlieues}~\guillemotright~--- or French
provinces even as far as bigger towns like Marseille --- not to
mention farming regions. These universities deliver an education
outside of the main curriculum, preparing their students rigorously
for their potential lives as the next politicians or company
directors\cite{grandes-ecoles-telegraph}. The question could be asked,
having learned about these prestigious universities as to whether the
principle of equality is completely destroyed and has been mocked,
given thier selective nature (only approximately 5\% of pupils who
enrol are eventually admitted\cite{grandes-ecoles-telegraph}) --- they
are very much exclusive and not open to all. Graduates from these
universities, as against standard and public universities, were mostly
educated in big, prosperous and rich cities such as Paris. A popular
feeder school is the Parisian Institute for Political Studies, which
in itself feeds from many countries\cite{sciences-po-admissions}, but
only the brightest and best students --- thereby increasing the
elitism and deflating those who may aspire but not be far enough up
the social ladder.\\

Relatedly, it is important to note that the French pride in their
school system is fairly justified --- in these modern times, it has
one of the highest attainment to educational spending ratios when
pitted against other European countries. According to the 2011
Organisation for Economic Cooperation and Development (hereafter
referred to as OECD) European educational attainment
report\cite{oecd-2011}, France spent \euro{122,429 million} on their
education system. It ranks 25th for attainment in mathematics
according to the Daily Telegraph which in turn cited the 2013 OECD
report\cite{oecd-2013}, above the United Kingdom which only spent
\pounds 94,801 million in 2011\cite{oecd-2011}, approximately an
equivalent of \euro{121,008} at today's rates.\\

Antoine Prost wrote about the contribution of the Republican primary
school to the concept of French national identity and therefore
national values in his book of essays entitled `Republican Identities
in War and Peace' that focused on the 19th and 20th
centuries\cite{prost}.  Prost obviously had an insight and reflection
on the post-War French education system and the role it played after
the War. This structure in the time of unrest post-War, when no-one
really knew what France was anymore and what to do, was due to the
surprising strength of the Republican system prior to the War. The
school system of egalitarianism and patriotism brought about by the
Third Republic of France, which had just come into play in 1870, was
instrumental.  Patriotism, or
\guillemotleft~\textit{patrie}~\guillemotright, is another of France's
very strong values, especially in schools\cite{patriotism-schools},
even though today it is not officially included in its three core
values per se. The French, even these days, are very proud to be
French\cite{proud-french}, as they were before and during the
War\cite{proud-war-french} --- the latter because the country held
together under the leadership of the Colonel Charles de Gaulle, even
when he fled France for London.\\

The French Republican government and society, especially the Third
Republic which came into force in 1870, was of course not at all times
a bed of roses. Adolfe Thiers was a leading light in the adoption of
this new Republic to govern France, so much so that he became its
President in its first incarnation of government. He was not very
enamoured by his choice --- indeed, he referred to it as a compromise:
``the form of government that divides France the
least''\cite[p.~351]{world-civilisation}. Although liberty, freedom,
secularism, equality and brotherhood were all part of the French
political values and school system back then, come 1914 the Republican
leadership started to crumble. With its rocky start it could be
argued, such as by \citeauthor{republican-experiment} in his book `The
Republican Experiment' in \citeyear{republican-experiment}, that
Republicanism was absolutely not ready for the mainstream, but just an
experiment (as the book title suggests). However, France could not
have survived without a government back then --- if it had, who would
have ordered the soldiers in the War? ---, and at least the Third
Republic's national principles were solid and comprehensible which led
to its seventy year reign as the chosen system of government ---
longer than any other system of
government\cite[p.~15]{republic-seventy}.\\

I mentioned above the Colonel Charles de Gaulle. His influence only
really became a focal point in the years of war, therefore 1914
onwards, but his way of managing and thinking had been developed over
the years of him being in the army during the 1800s. He had been
involved in the Franco--Prussian War which was the war that waged at
the time of the creation of the aforementioned Third Republic of
France, and he had strong views on how the army should conduct itself
in line with France's national values.\\

Back onto the topic of schools, I referred to patriotism above. All
was not always completely well in the administration of French schools
in the 1870s. A paper by \citeauthor{republican-school-pressure} in
\citeyear{republican-school-pressure} states that although the
education system aimed to ``[fight] inherited disadvantage and [lift]
children above the poverty line through state-based education,''
inequality was still rife despite very admirable principles and laws
introduced by the Third Republic. This situation has arguably got
worse in modern times\cite{higher-education-modern}, especially when
immigration and greater socioeconomic divides between regions of
France are taken into account. Jules Ferry's education policy
introduced in the 1870s took hold rapidly and successfully, leading to
the masses (such as the inhabitants of the poorer regions of France)
being educated and therefore learning to read, enabling them to feel a
greater sense of Frenchness and therefore belonging through
propaganda. This furthering of the people's education was definitely a
furthering of France's egalitarian values --- narrowing the gap
between rich and poor, and all in the same, founding way: through
basic school instruction and social normality.\\

Attention must be given to the school system of the French colonies,
too. During the reign of the Third Republic, yet before 1914, France
colonised Madagascar and Algiers, among other countries, and made
considerable efforts to standardise their national values in those
countries. They made state schools free and every child was to be
admissable to them regardless of their social status, race, or
linguistic ability\cite[p.~26]{colony-schools}. After all, school is a
place for learning. This was done to identify the colonies as part of
France too, as was important for building the Empire and lining
everyone up in accordance with the French national values of liberty,
equality and fraternity (or brotherhood). Widening the adoption of
these values thereby furthered the cause to form
\guillemotleft~\textit{Françafrique}~\guillemotright. However, France
itself was not ahead of the game when it came to introducing these
principles into their education system. As once again discussed in
\citeauthor{colony-schools}'s book on page 26, French Carribean
islands such as Haiti implemented the free and equal primary education
policy ten years before France introduced and mandated it.\\

An important consideration which in separating the State from the
Church went largely unnoticed was inclusivity of those of a religious
disposition. Be they Muslims from the newly colonised Maghreb area of
Africa, Christians ``indigenous'' to France, or Jews, teaching about
and therefore widespread understanding and considered acceptance or
rejection of these religions did not (and still in 2014 does not)
happen in schools.  Some may argue that this goes against the national
value of equality through not teaching religious tolerance on a
prescribed, national level, and they may be right. Some, however, such
as \citeauthor{secularism-french-religion} in his BBC News article
`The Deep Roots of French Secularism', even consider secularism to be
``the closest thing [France has] to a[n officially recognised] state
religion''. It is indeed possible that it might be the foundations of
equality through not being discriminatory against one religion, but
all religions. This is evident in modern times where neither Muslim
headscarves nor Christian crosses are allowed to be worn by anyone
working in public services or receiving state-provided education (also
\citeauthor{secularism-french-religion},
\citeyear{secularism-french-religion}). France's absence of religion,
even back in the 1900s, did not mean that all religious activity or
education stopped. If that had happened, how would the majority of
citizens have got through the then-impending War? Stoicism, or simply
indifference, was not that popular, joyous, or any kind of realistic
option, despite what philosophers such as Epictetus tried to pass on.
Even though all of France's state schools are secular, Catholic
schools for example have made an appearance across the country,
charging fees due to them only being partially subsidised (at the
level of the teacher's salaries) by the State. After all, the State
cannot be seen not to give any aid to religious groups for fear of
seeming intolerant in itself --- the exact thing it is trying to stamp
out with its general ``teaching'' of secularism, obviously not to be
confused for a nationally mandated ignorance towards religion.\\

In conclusion, the French school system over the years 1870 to 1914
laid the foundations for today's French school system, but those
foundations were laid as if they were houses on flood plains: prone to
torrents not of water, but of criticism. This criticism could be
partly caused by the Government's excessive political control and
uniformity of children's learning, indeed a championing cause that had
many sensible reasons for it coming into being. However, it did unite
and institutionalise France's values, and this was an outcome that
proved to be instrumental in uniting France by giving it its identity
back in the pre- and post-War years.

\bibliographystyle{plainnat}
\bibliography{second_assignment}

\end{document}
