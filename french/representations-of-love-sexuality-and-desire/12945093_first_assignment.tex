\documentclass[12pt,a4paper]{article}

\usepackage{fullpage}
\widowpenalty=100000
\clubpenalty=100000
\setlength{\parskip}{1ex plus 1ex minus 1ex}

\begin{document}

\begin{center}
    \subsection*{``They're all murderers, Dorsday and Cissy and Paul. Fred
				 is a murderer too, and Mother is a murderess. They've all
				 murdered me and pretend to know nothing about it. She killed
			     herself, they'll say. You've killed me, all of you, all of
				 you.''\\
				 Discuss this quotation from Schnitzler's Fr\"{a}ulein Else
				 with close reference to the text.}
	\subsubsection*{Isabell Long (12945093)}
	\subsubsection*{15th January 2017}
\end{center}

% Introduction.
Fr\"{a}ulein Else is a novella written by Arthur Schnitzler in 1924. It is written as though it were a stream of consciousness of Fr\"{a}ulein Else, the eponymous heroine. It is a story of how she is forced to prostitute herself to her father's old friend---Herr Dorsday---to save her family---particularly her father---from business and societal collapse due to gambling debts.\\

The quotation ``they're all murderers'' followed by the list of names---Dorsday, Cissy, Paul, Fred and Else's mother---shows not only all the people that Else feels killed her by not supporting her or, indeed, ``pretend[ing] to know nothing about it''. They did not directly murder her, but her plight could have been prevented by having some care shown to her. She admits on page eighteen that her parents sent her away to give her an education and piano lessons, but that she could not earn enough yet to help her father: ``[o]h God, why haven't I any money? Why haven't I earned anything yet? Why haven't I learnt anything?'' In this, even though confidence crises can happen at any age, the reader really grasps just how young Else is, and what a horrible position she is being put in. ``I can play the piano'', she muses, ``I've been to lectures on the history of art.'' Incredulously, she laughs ``ha ha!'' at this point. It is as if she comes to realise how little the pursuits of girls---dainty as they are---prepare her for the outside world, and hence that she must always defer to men in her society and social class. Her mother prostituting Else instead of herself is strange, as their only daughter, but it goes to show that her father cannot know about it, otherwise it is probable that he would have had something to say in the form of ``no'', if he is a decent man who, deep down, does care for his family and not just his gambling. So, the demands on Else are, it seems to the reader, purely her mother's last resort, although Else herself does not realise this, thinking on page 49 ``oh what he must have gone through before he made Mother write this letter!'' The fact that her mother goes on to say in this letter (page 16) that Else ``must not be angry with [them, her parents]'' and she assures Else's that ``there is no harm in it''. Of course, there is harm in it, for Dorsday wants his reward for giving her father the thirty thousand gulden, in the form of a naked girl to stare at. At this point, her mother is just preying on her daughter's nineteen year old naivety and the trust that mother and daughter have implicitly. Of course, Else is sceptical and that is why she is such a mess about the entire affair when it comes down to hearing Herr Dorsday's wishes of her.\\

One of the prevalent themes in the book is that of suicide. On page 68, Else tells of her ``powders'', exclaiming ``thank God I've got [them]!'' She starts to come to her senses, saying how she ``only wants to look at them''. This in itself is no worse than other people in the modern or not so modern age staring at lakes, or train tracks, or razor blades for self-harm, but like everything, it is a slippery slope. Else has clearly put a lot of thought into this, because she admits to having taken some: ``the day before yesterday I took a powder [...] ssh, don't tell anybody!'' This exclamation is interesting because Else is writing a diary, which are historically and traditionally private things---even these days diary packs for children are marketed as being secret. It is possible that Else would like someone to read her diary, to understand the plight and emotional torment she is going through, but that she is scared about word getting out to her family about her not wanting to do the deed, about her wanting to kill herself instead, even at the expense of her keep her family from harm or their impending collapse of social status. She appears to dare herself not to commit suicide---``here's the glass of veronal'', she states in her imaginary world where she is baring all to Dorsday on that fateful night that will come oh so soon. With ``Look, ladies and gentlemen'', she dares the audience---her friends and Dorsday's friends---to look at her as she takes the powder and falls down. She implores them to share in her torture, to stare at her in the eyes and put themselves through her torture at the same time that she is, and hence watch a girl kill herself. ``Now I take it to my hand. Now I raise it to my lips,'' she narrates as if in a dream-like state. ``At any moment I can be where there are no Dorsday and no father who misappropriates trust money.'' Again, she is daring herself to do it on that fateful night quite soon. She appears to resent her father, too, stating that he ``misappropriates trust money'', and the reader knows from the letter she received that that money goes into gambling. Indeed, while reading the letter all that time ago, she cries ``oh God, father, what have you done?''. It could be said that Else resents her father for not providing for the family, for thinking of himself before everyone else and leaving her mother no choice but to pimp her out to Dorsday to receive money to set their life back to how it should be. Earlier on at the end of page 51, she exclaims ``oh, how lovely it would be to be dead!'', carrying on to morbidly imagine the hearse that would pick her up, a ``really only nineteen'' year old, musing about what the others would think of the reasons she killed herself, a chief reason being that ``her father is in prison.'' Her father would only be in prison, of course, if she didn't keep up her end of the deal---allowing Dorsday what he requests---in order for him to provide the money. This proves that Else is not completely insensitive to her family's demands---but that was evident throughout the whole book---we can visibly see her tearing herself apart out of fear and loathing for Dorsday and anger at her mother for asking she do such a thing.\\

% Writing style.
Else's stream of consciousness style leaves little to the imagination in terms of what is going on at every step of the window we get into the goings on in this part of her life, but leaves the reader with vivid mental imagery at the same time. The novella does not have chapters, and indeed barely any punctuation apart from full stops and ellipses---the pages tend to run on. Thankfully for the reader, it is a short and compelling book, even if slightly gruesome, and so it is easily read in one sitting because one fears for Else's life and wants to know what happens next. Also, Schnitzler makes use of many rhetorical questions to illustrate how confused Else is at various points. The pages are made up of many short sentences, rather than long, rambling ones, too, which gives the reader the impression that it's depicting Else's current state, and that she is either very decisive or unsure about everything at the present moment.\\

In the last few pages of the book, Else enters delirium. This is indicated by the use of many ellipses and exclamation marks in the text. It is as though the reader is right there hyperventilating with her experiencing the horror as it appears in real life. ``Rascal, rascal! Here I stand naked!''---she dares Dorsday to come forward and do the unthinkable. ``Delicious thrills run over my skin'', she intimates, ``how wonderful it is to be naked!'' This implies liberation, or the thrill of doing something taboo---forbidden---for ladies of her class in society at the time. The adjective ``delicious'' here implies that the feeling was pleasant, maybe even that she had put aside her fears and her knowledge of right and wrong, and was just letting herself go, just enjoying the feelings. She was embracing the ``wonderful'' thrill---better than she could ever have imagined. Even though she is at the same time repulsed by the fact that Dorsday might touch her, Else wants Dorsday to touch her---the adrenalin is flowing, she wants to know what it feels like because, one can assume, she has never been naked in front of anyone before. ``His eyes are glowing,'' she reports. ``He opens his eyes wide. At last, he believes!'' Sadly, she now believes herself that because he has seen her naked, that is the last he wants. ``Father is saved!'' she exclaims. But once an act like that has been performed once, you can never forget it and go back to normal. Yes, her father might be saved, but she will live with shame for the rest of her life. And, indeed, at this point, Herr Dorsday is recovering from that scene and has not sent the money yet, so the deal is not done, her father is not saved after all. She realises this, and before she falls over ``shrieks'' to Dorsday that he ``must send the money at once!'' The use of this imperative implores him to do it, and in a veiled way hopes that he is a decent man who keeps his promises and is not just using her dire situation to his advantage.\\

Unfortunately for the reader, Else never relives what happens after these tumultuous events by writing about it. We never get to know what happens next---whether her father is saved or whether Dorsday does indeed want more before handing over the fifty thousand gulden. One hopes that despite the despicable acts he threatens to perform and requests of Else, Dorsday would overall be a decent man and not renege on his promise to pay her father.\\

% Conclusion.
In summary, Fr\"{a}ulein Else was killed by her family---at least metaphorically---because she was forced into doing something she, as a young woman, should not have had to do in order to save her and her family's social status. This backfired, however, because she bared all to Herr Dorsday and was ashamed of her actions and so died inside, leaving nothing but a shell---and what life is that? Surely her mother---the one who asked it of her---and father would have had to feel some remorse as well, even if her father had received the money he required to save himself and the family's livelihood.

\end{document}