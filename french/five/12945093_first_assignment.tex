\documentclass[12pt,a4paper]{article}

\usepackage[frenchb]{babel}
\usepackage[utf8]{inputenc}
\usepackage[T1]{fontenc}
\usepackage{titling}
\usepackage{fullpage}

\setlength{\droptitle}{-10em}

\begin{document}

\title{Faut-il souhaiter une société multiculturelle?}
\author{Isabell Long (12945093)}
\maketitle

Une société multiculturelle en est une qui se compose de plein
d'autres personnes du monde, s'ils sont différamment réligieux, de
différents races, ou par exemple s'ils viennent d'autres cultures ou
n'ont pas les mêmes habitudes que nous, par exemple les américains
comparés aux anglais.\\

Cette forme de société a beaucoup de convénients, notamment
l'ouverture des horizons des personnes pas seulement natives mais
aussi venant d'autres cultures. Ces personnes apprenent beaucoup sur
comment vivre ou ne pas vivre, et elles considèrent les expériences
des autres, bien ou pas. Cela mène au développement d'une sociète et
culture melangée, qui tire les bons de toutes les cultures et qui
regette les mauvaises comportements. Une société riche en cultures
aide généralement a promouvoir la tolérence, une chose indispensable
dans n'importe quelle société de nos jours et des pays modernes.
Pourtant, il y a toujours les personnes qui ne sont pas en accord et
qui incitent la haine et la violence, ce qui sont par définition
intolérantes.\\

Les écoles sont pour le plupart multiculturelles en Angleterre, mais
en France la laïcité n'aide pas tellement avec cela. Les élèves et
même les professeurs musulmans n'ont pas le droit de porter la voile,
ce qui n'adapte pas les jeunes aux différences parce qu'elles ne sont
pas visibles. Les jeunes qui veulent montrer leur réligion peuvent
donc se sentir opressés, ce qui inciterait peut-être à la haine plus
tard. Mais pourtant, comment savent-ils que c'est ce qu'ils veulent ou
peuvent faire, s'il n'y a pas cet exemple et ce n'est pas la norme ?\\

Où le multiculturalisme se déclenche est généralement au niveau des lois de différents pays. Les immigrants ou même les touristes venant d'autres pays sont généralement très content d'y être arrivés, mais ne s'attendent peut-être pas à devoir respecter autant d'autres lois, soient-elles plus fortes ou plus faibles que dans leurs pays d'origine. Mais, en voyagant, quelques uns diraient qu'on se catégorise d'une autre culture et alors les personnes peuvent s'isoler en ne s'intègrent pas dans le monoculture --- la culture principale --- de leur pays d'accueil. C'est important de se rappeler que sans société multiculturelle éventuellement, on n'aurait pas accès à la nourriture d'aujourd'hui --- päella, curry, steak, dont beaucoup de personnes aiment tellement --- ou, un exemple plus sérieux, l'avancement des médicaments ou médecins, parce que beaucoup viennent d'ailleurs.\\

Les êtres-humains sont multiculturels et donc la société est à peu près par defaut. Si tout le monde était exactement la même et cette règle s'imposait partout, on reviendrait très proche du régime Nazi du race aryenne. Même si l'on n'apprecie pas les immigrés complètement, on apprécie les differences des autres, de nos amis, par exemple, et alors cela est un exemple de multiculturalisme qu'il ne faut pas oublier. Les habitudes des gens à Paris comparés à ceux dans le sud de la France seront légèrement différents, mais c'est important --- le multiculturalisme est aussi une expression de notre individualité comme êtres vivants, et la vie serait très ennuyant sans cela. Alors oui, il faut souhaiter et célébrer le multiculturalisme, parce qu'il enrichit le monde.\\

\end{document}
