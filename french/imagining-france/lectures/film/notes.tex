\documentclass[12pt]{article}
\usepackage[utf8]{inputenc}
\usepackage{fullpage}

\begin{document}

\section*{Imagining France, semester two weeks 1--5, Film and the
Algerian War}

\subsection*{Muriel.}

Alain Renais:
\begin{itemize}
  \item{Started making films in the 1940s --- ones which were concentrated
        on the problem of suffering. Unrepresentable suffering.}
  \item{1963, Muriel was his first feature film; critical disaster;
        passing reference to the Algerian war was deemed "too soon".}
\end{itemize}

Fragmentation of society and family relations in this film --- can new
relationships emerge from a state of haunting disillusion?

\subsubsection*{What do you make of this story? Why is it so hard to
reconstitute?}

Disjointed, fragmented, hard to bring together, jumping from past to
future.

\subsubsection*{Who are the characters?}

Alphonse $=$ ex-boyfriend of Hélène, now married to Simone, supposedly
living in Algeria.\\
Hélène $=$ Bernhard's stepson.\\
Bernard $=$ national service in Algeria, stepson of Hélène, secretive,
highly-strung.\\
Françoise $=$ ``niece'' of Alphonse, actually his lover.

\subsubsection*{Do they themselves understand their past or their present?}

Bernard understands his past. He's ashamed. Doesn't understand his
future because he's too stuck in the past. Alphonse understands his
past, else he wouldn't be able to tell so many lies.

\subsubsection*{Why do the characters tell so many lies?}

The characters don't want to face the truth of their lives because
it's too hurtful. Alphonse wants to escape, hence the lies, to change
his present circumstances.

\subsubsection*{Why are there so many confusing minor characters?}

Lies affect many people --- ``web of lies''.

\subsection*{Caché.}

Enjoying the terrorising of middle class homeowners who previously led
a comfortable life? Pseudo-ethical terror?\\

Caché and Muriel are very similar --- they push the viewer over the edge
and hopes to break them - but Muriel does it at the end.\\

Response to horror is to film it, in the case of this film and many of
the director's others. Funny Games - another film by Hanneke ---
terrorised again. Characters here also called Ann(e|a) and George(s).
Piano Teacher is another.\\

Hanneke thinks that modern civilisation is locked into a false way of
behaving --- it's not able to react authentically or naturally to
catastrophe. What are the implications of this?\\

The film seems to be a critique of whether or not people feel things.
Where is actual pain or joy if our lives always resemble TV shows?

\end{document}
