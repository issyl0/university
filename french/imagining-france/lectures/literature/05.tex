\documentclass[12pt]{article}
\usepackage[utf8]{inputenc}
\usepackage{fullpage}
\title{IF05}

\begin{document}

\section*{Imagining France, Lecture 05, 2014-10-27}

\subsection*{Letters of a Peruvian Woman}

\subsubsection*{Could this be a story of the heroine's gradual progress towards self-affirmation and independence?}

\begin{itemize}
  \item{Yes, because she remains fiercely optimistic of finding Aza, and keeps her dignity in the face of sexual interest from men. She does not feel particularly bereft for being captured. She remains Perivuan even though she takes the opportunity to learn the French language.}
  \item{She is not fully integrated into the French culture and society --- this is her active choice. In the end, she gains confidence with the French language: she is able to write highly philosophical letters.}
  \item{She rejects Déterville's love, and therefore the stereotypical role of the French woman, that is, to be married. She stays true to her values, writing at one point that ``the passion of love is dangerous''.}
  \item{She missed out on a lot of nature's delights when she was in Peru, so she is happy to have experienced new things. She compares and criticises Peruvian culture, and this is brave.}
\end{itemize}

\subsubsection*{Is there an alternative, possibly regressive, interpretation in which she retreats from reality into dream and illusion?}

\begin{itemize}
  \item{Afraid of lying, she retreats from the possibility of being unfaithful with Déterville, she simply does not move on from Aza --- her vision is tunnelled.}
  \item{Déterville buys her a house from her money --- can she not make decisions anymore? Has she become irrational?}
  \item{In her house are all her possessions retreived by her captors --- she continues to live in the past as if France were Peru.}
  \item{Aza has adopted Spanish values --- he has married someone Spanish and do the values she is comparing herself and himself against really exist anymore?}
  \item{Is she writing to recreate an illusion, when she realises via Déterville that Aza isn't reading her letters? Why doesn't she just stop writing? Maybe she is trying to understand herself better, work out her life purpose, or relive the past in retreating to transcribe her knots into letters.}
\end{itemize}

\subsubsection*{Is there a feminist reading of the novel?}

\begin{itemize}
  \item{Experiences of different cultures and sexualisation in different cultures.}
  \item{She stays true to her beliefs and does not give in to Déterville's advances. She remains independent by choosing not to marry. Marriage is a choice --- not getting married is not particularly feminist, but the active decision is.}
  \item{Zilia decides to take control and master the French language through her hypercomplicated French letters. In these letters she touches on things that she perceives to be wrong about French society's treatment of women, such as sexualisation and women's dependence on men.}
  \item{Feminism is, however, a 20th Century concept. Is it fair to attempt to apply it to an 18th Century novel, where these concepts and preconceptions didn't exist? What is the modern definition of feminism anyway? The media portrays it as ``man hating''.}
\end{itemize}

\subsection*{How to approach a commentary.}

\begin{itemize}
  \item{Positive and negative of the passage.}
  \item{Examine literary devices and the structure of the passage.}
  \item{Quote from the passage to back up points.}
  \item{Meaning of the passage such as societal or historical context.}
\end{itemize}

\end{document}
