\documentclass[12pt,a4paper]{article}

\usepackage[cyr]{aeguill}
\usepackage[frenchb]{babel}
\usepackage[utf8]{inputenc}
\usepackage[T1]{fontenc}
\usepackage{titling}
\usepackage{fullpage}

\setlength{\droptitle}{-10em}

\begin{document}

\title{Comparaison de deux textes.}
\author{Isabell Long (12945093)}
\maketitle

Ici, j'analyserai deux textes. Le texte A, «~le mystérieux `homme au
piano'~», publié dans le journal \textit{Le Monde} en 2005, et le
texte B, «~Piano Man livre quelques secrets~» écrit dans le journal
\textit{Aujourd'hui en France}, aussi en 2005. Ces deux articles
traitent le même sujet --- un jeune homme qui a inventé des activités
qui étaient le raison pour son disparition, mais qui faisait semblant
de les avoir oublié --- «~amnésique~» --- mais au plus sérieux ---
«~maladie psychatrique~» --- lors de son retour chez lui en Allemagne.
Il est important de noter que ces deux journaux sont destinés pour de
differentes méthodes de lecture, et à de différentes types de
lecteurs.\\

Le Monde est un journal français très serieux qui existe depuis des
siècles, tandis qu'\textit{Aujourd'hui en France} est un peu moins
traditionnel en ce qui concerne son style et ton. Ces différences sont
apparents dès les premiers paragraphes. Le Monde, par exemple, utilise
beaucoup de questions rhétoriques pour faire penser le lecteur à ce
qui aurait pu se passer : «~Qui se cachait derrière [\ldots] ?~».
Comme Aujourd'hui en France s'est placé dans le cadre de reportage des
faits pour le plus grand public possible : le langage est plus simple,
même familier, et il n'y a rien de caché. On voit cela avec les mots
que l'auteur utilise : «~boulot~» au lieu de «~travail~». Les phrases
du deuxième texte sont plus courtes. \textit{Le Monde} essaye avec ses
phrases plus longues de faire plus une histoire, d'inclure plus
d'informations sur ce qui est arrivé à cet homme, parfois en spéculant
et avec l'inclusion des témoignages des journaux britanniques et
allemands (il est écrit qu'ils l'avaient pris la première fois pour un
«~étudiant irlandais~»).\\

Quant à la mise en page des deux textes, ils sont tous les deux
organisés en deux colonnes. Les deux textes aussi sont sous-titrés,
séparés en de différentes parties par de grands caractères italiques
dans le premier texte, et par une phrase en caractères gras et
majuscules dans le cas du deuxième texte. Il est souvent important,
peu importe le lectorat, d'organiser son texte, et l'utilisation des
sous-titres aide à le faire.\\

Une autre différence très marqué pour moi est que le texte A fait
référence au suicide sur la ligne 45, comparé à l'article
d'\textit{Aujourd'hui en France} qui ne l'écrit nulle part. On peut
spéculer que c'est à cause du lectorat étant moins influencé par ce
qu'ils lisent dans Le Monde, parce qu'ils sont plus vieux ou ont de
différentes idées, comparé aux jeunes qui liront peut-être le deuxième
journal avec plus d'aise.\\

Une omission significante des deux textes est une image de l'homme
duquel ils parlent. Ceci est peut-être parce qu'une image, que ce soit
une photographie ou un dessin, prendra plus de place sur la page et
ils n'auront pas la place de finir d'écrire tous ce qu'il voulaient.
Aussi, les deux articles sont remplis de beaucoup de description de
l'homme, et s'ils avaient mis une photographie ils n'auraient pas pu,
peut-être, prolonger l'histoire assez pour faire un vrai article. Il
est évident dans le texte B que cette histoire n'a pas été la plus
importante du jour, parce qu'elle a été mise dans la partie des faits
divers.

\end{document}
