\documentclass[12pt]{article}
\usepackage[utf8]{inputenc}
\usepackage[T1]{fontenc}
\usepackage{fullpage}
\title{F402}

\begin{document}

\section*{French 4, Seminar 02, 2014-10-08}

\subsection*{Oral presentation details}

19 novembre, entre 5 et 10 minutes de présentation, avec 5 minutes après pour des questions. Notre sujet de présentation mène à un débat général entre toute la classe, qui dure 15 minutes.

\subsection*{Debate}

Le fait que les partis politiques d'extrême droit existent est-il regrettable ? Sans les partis politiques d'extrême droit, que sera la démocratie ?

\subsection*{How to structure a commentary of an advert}

\begin{itemize}
  \item{Couleurs.}
  \item{Photographe ou dessin ?}
  \item{Ce sont des idées claires ou obscures ?}
  \item{L'image montre ce que la publicité essaye de vendre ?}
  \item{Les personnages dans la publicité---que font ils ?}
  \item{Le texte est-il téchnique ou facile à comprendre ?}
  \item{Est-ce qu'il existe des anglicismes dans le texte ?}
\end{itemize}

\subsection*{Analysing a piece of writing\\ Roland Barthes, Le Message Photographique (1961)}

\subsubsection*{\textit{Selon Barthes, en quoi consiste « la source emmetrice » d'une photo de presse ? S'agit-il d'une seule personne ?}}

Selon Barthes, « la source emmetrice » consiste de toutes les personnes qui ont fait le travail pour faire publier le journal.

\subsubsection*{\textit{Quels sont les entours d'une photo de presse ?}}

Les entours d'une photo de presse sont le texte, les publicités dans le journal, le titre, et pour donner du contexte, le nom du journal pour en avoir une idée de comment se passera l'analyse du sujet.

\subsubsection*{\textit{A part l'image elle-même, quels sont les facteurs qui peuvent infléchir la réception et l'interprétation d'une image photographique ?}}

La mise en page de l'image avec d'autres articles autour. Il y a souvent des liens entre les autres articles du journal. La taille du photo peut être important aussi.

\end{document}
