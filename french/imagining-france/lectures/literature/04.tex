\documentclass[12pt]{article}
\usepackage[utf8]{inputenc}
\usepackage{fullpage}
\title{IF04}

\begin{document}

\section*{Imagining France, Lecture 04, 2014-10-20}

\subsection*{Letters of a Peruvian Woman}

\subsubsection*{What does Letters of a Peruvian Woman have in common with L'Ingénu? What differences does it have?}

\begin{itemize}
  \item{Both stories are episodic --- both characters move around their worlds.}
  \item{Epistolary novel vs.\ short story.}
  \item{Male vs.\ female protagonists.}
  \item{In both texts there exists criticisms of French culture and hypocrisy.}
  \item{Philosophical writing is used as a form of societal and cultural criticism.}
  \item{One is in a letter format, the other in the usual format of a short story. The letters don't immediately tell a story.}
\end{itemize}

\subsubsection*{Which elements of French culture does Zilia criticise?}

\begin{itemize}
  \item{The stereotype of marriage: at the end of the novel she chooses friendship as opposed to marrying either Aza or Déterville.}
  \item{The treatment of women, for example their sexualisation and materialism and the fact that they all act the same and conform to societal norms.}
   \item{Peru is full of colour --- Zillia writes about light and sun and dark --- all these themes are interwoven into her letters. She misses the sun when she is in France, as she tended to stay indoors.}
\end{itemize}

\subsubsection*{How is pathos used for the purposes of social criticism in letters 13, 19 and 20?}

\paragraph{Letter 13:}
\begin{itemize}
  \item{Aristocratic family structures --- lack of love between the mother and her own children.}
  \item{The maid is kinder to Zilia than the rich.}
  \item{In Peru, the King helps his people whereas in France he does not.}
\end{itemize}

\paragraph{Letter 19:}
\begin{itemize}
  \item{Uncertainty of Déterville's departure --- Zillia has lost her bridge that connects her to Aza, and grief ensues.}
  \item{Déterville's mother does not want him to marry and wants to keep the family rich.}
\end{itemize}

\paragraph{Letter 20:}
\begin{itemize}
  \item{``I have told you nothing of the troubles in my mind, dearest Aza.''}
  \item{``Sorrows caused by the prevailing customs of this nation.''}
  \item{``The Government cannot fail to be imperfect.''}
\end{itemize}

\subsubsection*{In which ways is Zilia perceived as `other' by the French characters she meets?}

\begin{itemize}
  \item{Her clothes are different. She eventually chooses to wear culturally acceptable clothes, but is exposed then to sexual interest.}
  \item{In the beginning, she writes in knots as it is all she knows}
  \item{She does not speak or understand any French until quite a way into the novel. Her early letters use very flowery, metaphorical French.}
  \item{When she does speak, she speaks very directly, unlike French women who think carefully about what they say.}
\end{itemize}
\end{document}
