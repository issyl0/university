\documentclass[12pt,a4paper]{article}

\usepackage[frenchb]{babel}
\usepackage[utf8]{inputenc}
\usepackage[T1]{fontenc}
\usepackage{titling}
\usepackage{fullpage}

\setlength{\droptitle}{-10em}

\begin{document}

\title{La télévision est-elle une forme d'art ?}
\author{Isabell Long (12945093)}
\maketitle

La télévision est devenue presque indispensable dans notre société ---
un point central autour duquel la famille s'assoit d'une soirée pour
se réuinir. Beaucoup de personnes regardent les émissions de
télévision pour se reposer, d'autres pour apprendre des choses
différentes de ce qu'elles apprennent dans leurs vies quotidiennes au
travail, ou encore un mélange des deux.\\

Il y a beaucoup de travail à faire pour construire un drame, un série
policier, les infos du soir ou encore les bandes dessinées pour les
enfants : il faut que ces differents genres plaisent au public. Il y a
les acteurs ou les lecteurs, ceux qui font la mise-en-scène, et après
que les émissions ont été crées on a besoin de quelqu'un pour les
diffuser à la bonne heure.\\

La question de si les émissions sont de l'art ou pas depend fortement
de ce que l'on regarde. Les scientifiques en Angleterre --- par
exemple Brian Cox qui diffuse les séries comme \og{}Wonders of the
Universe\fg{} qui explique les éléments fontamentaux de notre
existence --- ne croient pas peut-être que leur profession devrait
être simplifié pour le grand nombre de personnes qui regardent les
chaines à 20h du soir --- tout cela au nom de rendre accessible la
science à tous pour faire profondément penser le public. Les
scientifiques résisteront peut-être cette transition à la science
populaire de leurs recherches et connaissances, en voulant les garder
entre personnes appréciées. Ces mêmes n'apprecieront pas la science
converti en \og{}art\fg{} pour que ce soit facile à regarder et
comprendre. Dans le cas de faciliter la compréhension à le plus grand
public possible, la mise-en-scène est très important. Par exemple,
très peu de personnes vont s'accrocher à un grand tableau de maths,
mais s'il est animé un peu avec les graphiques, ça devient plus léger.\\

Parlant ensuite des infos. Ils ne peuvent pas être considérés comme de
l'art, parce qu'ils jouent un grand rôle dans la société en informant
les citoyens des événements à la fois dans leur pays et à l'étranger,
et le fait qu'ils soient sérieux donne de la crédibilité aux
reportages. De l'autre côté, si les infos se diffusaient comme des
bandes annonce de film, par exemple la guerre en Afghanistan, plus de
gens les écouteraient sérieusement, peut-être.\\

Dans l'age où presque toutes les personnes dans notre société moderne
ont accès à Internet, certains pensent que l'Internet, avec sa
capacité illimité de stocker les connaissances, et le fait qu'il ne
coupe pas les pensées dès une heure pour avancer à la chose suivante,
surpassera la télévision à être le passe-temps favori. Mais les deux
choses ne sont pas très differentes --- elles peuvent travailler
ensemble pour reunir un autre catégorie de public, en leur laissant
exprimer d'une façon différente leurs idées à plus du monde, et en
affirmant aux autres l'importance de la télévision de provoquer la
discussion, que les émissions soient de l'art ou pas. Il faut dire que
Banksy, l'artiste anonyme, ne serait pas dévenu aussi connu si
l'Internet et la télévision n'auraient pas travaillés ensemble pour
faire circuler son message, ceci étant bien sur de l'art.\\

L'art dans les émissions est un peu cela : pouvoir deviner comment
présenter les choses pour qu'elles touchent le plus grand public
possible, et la télévision a eue du succès.

\end{document}
