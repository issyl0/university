\documentclass[12pt]{article}

\usepackage[numbers]{natbib}
\usepackage[frenchb]{babel}
\usepackage[utf8]{inputenc}
\usepackage[T1]{fontenc}
\usepackage{url}
\usepackage{fullpage}
\usepackage{breakurl}
\usepackage[breaklinks]{hyperref}
\usepackage{titling}

\begin{document}

\begin{center}
	\section*{Project Proposal -- Isabell Long (12945093)\\
	    			Immigration France--Angleterre\slash Angleterre--France :
						pourquoi il y a eu tant dans les années '00?}
\end{center}

% What will I research about it? What is my hypothesis?
L'immigration est un phénomène très important de nos jours. Dans ce
travail, je propose de rechercher \textit{pourquoi} ou même
\textit{si}, dans les années `00 (de nos jours) plus que les années
`90 ou `80, le taux d'immigration en France a augmenté,
particulièrement le taux d'immigration des anglais. En particulier
avec les réfugiés en ce moment, je voulais étudier pourquoi les
personnes choisissent vraiment de déménager, de bouscouler leurs vies
quand elles \textit{veulent} le faire, non pas quand elles
\textit{sont poussées à le faire, n'ayant plus le choix, à cause
d'être désespérées}. Je me concentre sur l'immigration des pays
européens parce que, avec les réfugiés par exemple, mon projet
pourrait devenir très grand et aller facilement hors sujet. Je
généralise peut-être un peu, mais cela est à voir quand j'aurais fait
mes recherches.\\

\noindent Il existe beaucoup de statistiques à propos (voir en bas).
J'utiliserai les chiffres et justifications citées dans
les références en dessous. Aussi, je redigerai et j'envoyerai un
sondage à des personnes qui ont déménagées en France. Il est important
de voir les deux côtés pour valider mon hypothèse, alors j'envoyerai
les questions à des personnes qui sont revenues, aussi. Je chercherai
des personnes en écrivant sur twitter et facebook. Répondre aux
questions sera complètement anonyme et il sera le choix des personnes
de répondre ou pas. Il n'y aura pas de recompense (par exemple, du
chocolat, de l'argent) à la fin, et je supprimerai les réponses après.\\

% What other areas of interest/academic disciplines does it touch on?
\noindent Mes recherches toucheront à la sociologie, à la politique,
l'histoire-géographie, et bien sûr, la mode et joie de vie comparée à
ce que ces personnes ont laissées derrière elles, mais en pensant
qu'aux européens, à la France métropolitaine et à l'Angleterre.

\nocite{europa-migration-statistics,oecd,ined,govuk,immigration-1986,
				sayad2004suffering,original-questionnaire}

\bibliographystyle{plainnat}
\bibliography{research_project}

\end{document}
