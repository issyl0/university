\documentclass[12pt]{article}

\usepackage{fullpage}
\usepackage[round]{natbib}
\usepackage{url}
\usepackage{titling}

\setlength{\droptitle}{-10em}
\linespread{1.3}

\begin{document}

\title{\small{Is British Sign Language a language? Discuss with
              reference to sign structure and consider sociolinguistic
              variables.}}
\author{Isabell Long}
\maketitle

British Sign Language (BSL) is used as the preferred sign language ---
as opposed to American Sign Language (ASL) --- for many of the deaf
inhabitants of this country. According to the 2011 British
census\nocite{2011-census}, 22~000 people reported using a sign
language as their main language, of which 15~000 respondents specified
that the type was British Sign Language.\\

The Oxford English Dictionary definition of ``language'' states that a
language is a ``method of human communication, either spoken or
written, consisting of the use of words in a structured and
conventional way''. With this definition of what languages are and
are not, some could argue that written English qualifies as a separate
language to spoken English, or that American English is different to
British English or Indian English also in that regard. These are more
likely, however, to just be considered to be dialects, such as
`Northern' English vs.\ `Southern', or `Queen's' English. Dialects are
formally defined as ``particular form of a language which is peculiar
to a specific region or social group'' \citep{oed-dialect}.\\

In relation to sign language, it is interesting to note that nowhere
in the definition of ``language'' is mentioned the use of signs or
non-verbal communication apart from writing, such as body-language.
This could be seen as deliberately exclusionary --- suggesting that
sign languages in general are the underclasses, because potentially
some may view hearing people as superior. To take a less alarmist
view, many dictionary definitions are very old and British Sign
Language was only recognised by the UK Government as a strong,
bona fide, independent language in March 2003 \citep{gov-guidance-bsl},
though it still is classified as having minority status.\\

Unlike other minority languages such as Gaelic or Welsh, British Sign
Language is not taught in schools in this country. Fairly, Gaelic and
Welsh are only taught in Ireland (or Scotland), or Wales respectively,
where those languages are still actively translated into for the older
population, or still have importance for keeping heritage alive and as
a marker of an Irish, Scottish or Welsh person's identity
\citep{welsh-language-identity}. However, not learning these two
languages does not have an active impact on someone's ability to
communicate in the main language of the land in these modern times, as
in these cases it is English. This is quite often not the case for
deaf or hard of hearing people, unless they have developed their
deafness through unfortunate circumstances later in life after having
learned to speak, read and write in their `common' language
\citep{deaf-writing}. Some indigenous, minority, languages such as
Welsh or Gaelic are taught at GCSE level in certain parts of the
country where they are still relevant, and some campaign groups have
argued that therefore it is only right that British Sign Language be
too, only all over the country because all deaf people do not live
together in one part of the UK \citep{campaign-bsl-gcse}. They argue
that doing so would encourage deaf and hard of hearing students to
perservere with education in mixed schools, potentially enable them to
not feel so excluded from their peer group at a critical time in their
life, while at the same time enabling greater understanding and
awareness amongst the hearing population of the difficulties that deaf
people face and increase the amount of communication that goes on
\citep{campaign-bsl-gcse-bbc}.\\

With reference to sign language learning, people have recognised that
everyone has to start somewhere, hence the introduction of
fingerspelling --- the ability to spell out words just by signing
individual letters without having to know the specific sign for the
word you wish to use \citep{fingerspelling}. This has decreased in
popularity over the last ten years (also \cite{fingerspelling}), but
goes to show the variation in the way signers communicate up and down
the country, and with varying levels of confidence, such as in
standard English in this country there being different accents such as
those of the south and those of the north.\\

British Sign Language has a clearly defined structure in its grammar,
yet also the space in front of the body in which the signs are
performed, as well as the shape of the hands used when making the sign
and subsequently their movement, and whether the hands have, for
example, palms facing upwards or downwards \citep[p.\
408]{sociolinguistics-mesthrie}. Sign language users also make
heavy use of mouth movements and nodding \citep[p.\
3]{linguistics-of-bsl} which give context to the people communicating.
There is more written about mouth movements in American Sign Language
than British Sign Language, but it cannot be assumed that all of the
same principles apply given that the two languages are not mutually
intelligible \citep{ucl-sign-language-introduction}. We can, however,
compare their two different approaches to signing simple words and the
extra gestures (movements of the eyebrows or shoulders) that accompany
them both in American and British Sign Language respectively. To give
an example particular to the former --- American Sign Language --- we
turn to \citeauthor{sociolinguistics-mesthrie},
\citeyear{sociolinguistics-mesthrie}, page 409. He explains,
referencing some other research by \citeauthor{asl-facial-signals} in
\citeyear{asl-facial-signals} that questions with a response of
``yes'' or ``no'' are communicated through a ``raising of eyebrows'',
``forward leaning shoulders'' and ``wide eyes''. Questions that mean
to gain the questioner information are indicated with ``squinted
eyebrows'' and a ``forward movement of the head''. Therefore, the
importance of these head movements and facial expressions, as well as
mouth movements, go someway to showing why signing space is crucial.
If someone signs below their waist and is also performing facially,
this might be hard for the other person in the conversation to follow,
and this might not be in the signing vocabulary of many signers.
Related to this, to sign the British supermarket chain \textit{ASDA}
in conversation, signers tap their hip or thigh rather than their
buttocks, because that is more visible and there is no established
sign for behind the body \citep[p.\ 5]{linguistics-of-bsl}. The
adaptation of gestures to different contexts, people, sentences or
word formations is done also in standard English and in other types of
sign language, for example Auslan, the Australian sign language. For
example in standard spoken English it is normal to gesticulate to
intensify your point, as well as raise your voice. For signers, voices
are obviously not heard as a matter of course, but they still use
gestures and hand movements to emphasize or refer to different classes
of words, e.g.\ adverbs \citep[p.\ 86]{linguistics-of-bsl}.\\

To give another example this time related to mouth movement as opposed
to movement of physical facial features, the mouth moves to perform
the action for words such as ``vomit'' or ``laugh'', while the hands
perform the sign \citep[p.\ 84]{linguistics-of-bsl}. The above two
paragraphs show that all languages have idiosyncracies, and that they
are part of what makes a language a language. Sign language's sentence
structure is different to that of standard English. Again using an
example of American Sign Language, its sentences are constructed using
the Subject Verb Object (SVO) scheme. Hence we examine the sentence
``the dog chased the cat'': where in standard English this sentence is
ordered by \textit{`dog' --- subject, `chased' --- verb, `cat' ---
object} and that sentence is in the past tense, in sign language it is
signed as ``dog chase cat'', with the signer having knowledge of their
native language's grammar and knowing that that sentence, even without
a past participle `chase\textit{d}' is grammatical because of other
contextual clues \citep{linguistics-of-bsl}.\\

All languages evolve and sign languages --- whichever variant --- is
no exception to this. New signs are created all the time, especially
in the age of the Internet, and deaf people therefore need to know how
to sign ``photobomb'' or ``selfie'', to name some examples. These were
examined by Bill Vickers, a sign language teacher who himself is deaf
and has a vested interest in keeping sign language current, in
relation to American Sign Language \citep{asl-internet-slang}. There
is a lot of variation amongst old words, let alone new words. In this
experiment, two signers of different ages who self-identified as part
of the Deaf community (and who are also audiologically deaf) were
asked to sign these new words as they would normally do in
conversation, but on camera at the same time. They were reasonably
similar in concept, but the differences came in the specific actions
performed. These differences, however subtle, could, the reader
supposes, depend on age, whether they'd come to using the sign before
in conversation in a non-simulated situation, or general evolution of
language. Indeed, a lot of the comments from the research participants
in this study suggested that they hadn't used these signs before, and
made them up on the spot based on other words and their own thoughts
about what things like ``selfies'' are. This is a perfect example of
language evolution as in standard English --- after all, we
invented the word ``selfie'' and made it exist in the Oxford English
Dictionary in June 2014.\\

The statistics which were quoted in the introduction go to show that
the Census every five years is evolving and becoming more inclusive,
like languages themselves, and having been around for years, sign
language is becoming more and more recognised as a language as the years
go on.

\bibliographystyle{plainnat}
\bibliography{second_question_e}

\end{document}
