\documentclass[12pt]{article}
\usepackage[utf8]{inputenc}
\usepackage{fullpage}
\title{IF02}

\begin{document}

\section*{Imagining France, Lecture 02, 2014-10-06}

\subsection*{Initial thoughts about l'Ingénu}

\begin{itemize}
  \item{I quite liked it---it was humourous in places.}
  \item{He realised he was French but also a Canadian and had spent time living in England---a true international, with an eye for history.}
\end{itemize}

\textsc{For essays:} take account of the sometimes comic tone: how does Voltaire convey messages?

\subsection*{History of Voltaire and the period}

Important free-thinker. Many philosophical texts. Famous for his eighteenth century tragedies: it's only lately that his short stories have gained some value. Overarching themes of \textit{pathos} and \textit{irony}.

\subsection*{Class discussion topics}

\subsubsection*{\textit{As a Huron, is l'Ingénu given cultural specificity?}}

A lot is said about where he travels: many places, he is a Huron of Canadian descent but lived in England and speaks fluent French. He has a lot of historical context from reading books, so he can be said to not really have any nationality, just to be content and discovering. The French, such as Mlle.\ de St.\ Yves, take him in as he seems to be one of their own, but he is dismayed how slow the French are at battling the English, for example.

\subsubsection*{\textit{What qualities are ascribed to l'Ingénu?}}

\begin{itemize}
  \item{Reactivity: due to being an outsider he is not restricted to one point of view, able to see multiple sides.}
  \item{Brave; Frank: says things how they are, almost childlike.}
  \item{Conscient; recognises his own errors such as the naivety of wanting to follow the Bible by the letter and be baptised naked in a river, which went outside of social norms.}
  \item{Well learned: reads many books and has assorted interests.}
  \item{Curious blank slate in the beginning: shows intellectual humility, used to criticise others' points of view. Embodies the ideals of the Enlightenment.}
\end{itemize}

\subsubsection*{\textit{Why and how does l'Ingénu's position as a cultural outsider make him a good observer and judge of French culture?}}

L'Ingénu is a completely blank slate, therefore he is not concerned with aligning his views with society's for the sake of politeness. He enables revision of the reader's perspective on human nature and society.

\subsubsection*{\textit{What other nations or countries are evoked over the course of the text?}}

\begin{itemize}
  \item{Canada -- Hurons are his own tribe.}
  \item{England.}
  \item{France -- money and then philosophy and history and cultures.}
  \item{Christianity, which he is unfamiliar with -- Jesuits, Hugenots, Catholics, Protestants.}
  \item{Chinese and Roman -- he reads history books and questions whether France really is the centre of the world.}
\end{itemize}

\subsubsection*{\textit{What are the different elements of French culture that are criticised or satirised in this story?}}

\begin{itemize}
  \item{Christianity -- why don't the French religious people follow the Bible to the letter, i.e. being baptised in a river, or the priest not reciprocating by confessing his sins to the Huron.}
  \item{Political hierarchy -- the Huron was confused by not being able to directly speak to the King, he had to go to his ministers first.}
  \item{`French is the best language' -- the cultural superiority of the French -- not open to seeing beauty in other languages.}
  \item{Satire -- sacrements, aspects of doctrine, the Pope's position as the Head of the Church and societally respected.}
  \item{`Meaninglessness' when talking to priest in the prison. Why is the priest in prison if God is meant to protect people?}
  \item{Dishonesty -- marriage needs witnesses -- the Huron doesn't understand the pomp and circumstance -- if you love someone, you should just get on and marry them, there shouldn't be a justification. This ties in with the concept of human happiness -- should it not be one's own responsibility to decide one's own happiness? Forced marriage, going against social norms with the Huron loving Mlle.\ St.\ Yves?}
  \item{Corruption -- corruption of court, abuse of power of the court official forcing Mlle.\ St.\ Yves to sleep with him in order to free the Huron; bribes; counter-culture; injustice for the Huron with his arbitary arrest with no evidence.}
\end{itemize}

\subsubsection*{\textit{Find examples of how irony and pathos are used for satirical purposes.}}

\begin{itemize}
  \item{``She promised me marriage and yet will not marry me'' -- the Huron was confused by the different definitions of `honour' -- his code of honour is to do what feels right, marry someone you love, whereas hers is to not disobey society's rules.}
  \item{``Bathed in his aunt's tears'' -- this shows \textit{pathos} because he does not understand that he cannot marry his cousin, yet this is an example of \textit{dramatic irony} because the aunt does understand and therefore the reader knows something that the lead character does not.}
  \item{Statirising women -- women are chaste but still like to watch L'Ingénu naked in the river.}
  \item{Exaggeration, humour in the first chapter: ``[has she been] eaten by cannibals''.}
  \item{Mlle.\ St.\ Yves dies of shame, in effect, after sex with the court master in order to free the Huron. She falls gravely ill and the doctors don't look after her, they are more concerned with bettering themselves over other doctors. This ironises social attitudes towards Mlle.\ St.\ Yves' death -- she died of shame -- even though she had the Huron free and by her side, she could not live with her conscience -- society should not cause that to happen.}
\end{itemize}

\end{document}
