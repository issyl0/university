\documentclass[12pt,a4paper]{article}

\usepackage[utf8]{inputenc}
\usepackage[T1]{fontenc}
\usepackage[numbers]{natbib}
\usepackage{titling}
\usepackage{fullpage}
\usepackage{url}
\usepackage{breakurl}
\usepackage[breaklinks]{hyperref}

\setlength{\droptitle}{-10em}

\begin{document}

\title{To what extent can contemporary French cinematic
       representations of technology be said to be in any sense
       realistic? Discuss with reference to at least two films.}
\author{Isabell Long (12945093)}
\maketitle

The portrayal of technology in film is interesting in this day and
age, and compared to previous decades, because of the modern day
person's reliance on technology to perform their day-to-day tasks. In
this essay I will attempt to examine the realism of any type of
technology (with particular focus on computing technology) as
portrayed in French films in particular, as France could be said to
have been stuck in the dark ages with regard to its sometimes slow
adoption of mainstream technology. I will pay particular attention to
two French films: \textit{Demonlover} (2002), directed by Olivier
Assayas, and \textit{The Fifth Element} (1997) which was directed by
Luc Besson and stars Bruce Willis.\\

The grasp technology has on people's everyday lives varies,
understandably, from person to person. We know from commercial success
and following, or our own experience, that a film and in particular a
story has to be able to be related to society by its viewers. If it is
not, it neither fulfil its audience, nor sufficiently connect them
with the topic and storyline and therefore does not seem to have any
chance of becoming a positvely termed cult film. Therefore, it can be
said in many cases to be more important that a film has so-called
shock value, and a memorable storyline, than any attempt at realism.
Obviously, it depends on the type, style and genre of film.
Documentaries allegedly portray reality or a historical moment, but to
do this while captivating the audience means that artistic license,
inventing fiction to enhance the power of fact, is sometimes required.
This could be said to be what film does in its portrayal of
technology, particularly in hacker-style films such as
\textit{Hackers} (1995) with its fast-scrolling green on black text
and characters hitting the keyboard too fast to make sense of, or
\textit{James Bond: Skyfall} (2012) which has the protagonist being
tied up in a visually realistic yet aurally unrealistic, deathly quiet
server room.\\

% Why people feel the need to sensationalise technology - a lot of the
% public already don't use it to its full potential, and with this
% treatment of it in media, the general public's relationship with
% technology is unlikely to improve - they see it as "not for them"
% and "nerdy".
Films that include representations of technology often tend to overdo
them or dumb them down, making them unrealistic and sensationalist.
This can be disappointing to many technologically-minded people
because the general public tend not to be massively literate when it
comes to computers or digital products in general. A 2013 paper by the
Organisation for Economic Co-operation and Development (OECD, present
in thirty-four countries), entitled ``Skills Outlook'', stated that
France ranked below the average for its adults' literacy skills,
taking into account also digital literacy. ``Below the average`` was
far behind Japan who was the leader with a score in the hundreds, yet
also below even the United Kingdom which ranked about average. With
science fiction and technology-focused films becoming ever-popular
(McLean, the Guardian, 2007) with the rise of Internet of Things
technology such as mobile-controlled heating thermostats like the
Nest, plus smart weighing scales that provide real-time data to your
phone for statistical purposes for example, things thought of as
impossible are getting nearer every day to becoming possible. With
these advances, the general public are becoming generally accustomed
to having technology ``interfere'', or in more kind and rational
terms, ``fit into'' their daily lives.\\

The basic premise of \textit{The Fifth Element} is that of ``a cab
driver [who] unwittingly becomes the central figure in the search for
a cosmic weapon to keep [evil] away'' (IMDB synopsis, 2016).  First,
it seems a compelling science fiction film, due to the presence of
``cosmic weapons'' which don't exist in real-life and add another
sensationalist storyline or event to the viewer's consciousness. This
is frequently the point of anything sci-fi or any kind of fiction,
whether it be a book or a film: to take the reader or viewer to
another land: one of escapism from the harsh realities of life.
(Evans, 2001)\\

Trailers of films are a good way for viewers to see that they appeal
to them before watching them in the cinema or buying a DVD.
\textit{The Fifth Element}'s trailer is short, punchy, includes shots
of the main character Korben Dallas---Bruce Willis---in Bruce's
default state (based on every film he stars in) of barely dressed and
being asked to save the world. However, in terms of realism and
technology, the plot thins. The central elements are fire, water,
Earth and air, but these, especially Earth's survival, on which all
except fire depend---the Sun was, after all, formed out of the same
giant fireball of gases as the Earth (Redd, \textit{Space.com},
2013)---are threatened. Luckily for humans trapped in the clutches of
an evil takeover, there is a fifth element, a mysterious machine that
comes to earth every five thousand years to relay more of the central
elements and keep it ticking over. In this film, the fifth element's
travel to earth is sabotaged by so-called evils who are intent on
Earth's destruction. That in itself is an inconsistency and plot hole,
because if it happens every five thousand years, no-one could have
lived long enough to see it happen before and at this point, no-one
would know to sabotage something that has had no visibility in the
past. How would the evils know they had got the right spaceship?
Theoretically, there would have been many spaceships, or hovercars,
moving around, as it was a big city. Also in the trailer, the almost
Egyptian, tomb-like, structure surrounding the so-called fifth
element---a triangular shaped, gold-plated ship-type object with
flares as legs---is only opened with one key every five thousand years
to reveal no erosion, no sticky locks, and no worn away hieroglyphs or
missing pieces. This in itself is sensationalist, because research by
Burke and Gunnell in 2008 has proven that erosion happens over the
years even without human interference, meaning that even computer
controlled environments would still see some erosion and loss of
function, especially over five thousand years, especially with the
lack of digital technicians or electronic engineers in these fictional
worlds---everyone who has ever used a computer knows that they break
down at the most inopportune, inconvenient moments.\\

The story of \textit{Demonlover} seems on the surface quite a bit
darker and more disturbing while still not technically being classed
as a horror film: ``[t]wo corporations compete for illicit 3D manga
pornography, sending spies to infiltrate each other's operations.''
(IMDB synopsis, 2016) Thankfully for many viewers (I would imagine) it
is more concerned with the rival corporations' tactics than the
contents of the ``treasure'' they seek.\\

Demonlover's trailer is loud, shocking and seems very fast-paced and
glitchy. There are lots of split-second shots with disturbing sounds
of screaming, interspersed with blank screens, which seek to make the
viewer absorb what just happened without going at full speed into the
next hard-hitting sequence, and maybe even split up the film into
themes, leading to viewer intrigue. Unlike \textit{The Fifth Element},
the trailer does allude to its French director and therefore French
origins, having in the English version subtitles for the two sentences
that the characters say in French when conversing about the rival
company and their plans. It is also very colourful, as Japanese manga
is as it is the pop culture symbol of an entire country which is
colourful in itself (Sugimoto, 2014), and this adds even more to the
intrigue---the viewer is left wondering what the colours mean, if
anything, and whether these represent computers connecting to illicit
sites, as in obfuscating the workings of the internet, but this
comparison could be seen as tenuous or far-fetched.\\

The plot of \textit{Demonlover} is interesting because it centers
around acquisition of and profiting from an online pornography site,
the fictional ``demonlover.com''. Therefore, it covers explicit themes
such as those of pornography, cultural norms, and addiction. This film
was made in 2002, a little while before such strict controls and
regulations about broadcasting torture came into force, and thus it
also portrays the fights between rival corporations in the industries,
where people will stop at nothing to push creative boundaries and
profits. There are scenes of computer hacking a while into the film as
Diane infiltrates the rival company representative's hotel room and
attempts to download confidential files onto a USB stick to steal them
for her company's advantage, but this is done at speed and in a way
that many real computer hackers think of as too risky: in person.
They prefer keyloggers downloaded as viruses that transfer data as and
when it is captured whilst not relying on humans who, from various
events both fictionally and in real life in the past, are known to be
fallible pawns in the machinery of corporate espionage. Indeed, Diane
ends up fighting when the woman she spies on returns, yet after this
bloodshed she wakes up later without a mark on her. This has parallels
able to be drawn with video games, which the film also tries to
include in its screenplay: people die but it's not real, it's a
blurring of reality and fiction, you can shoot people but they get
back up again as good as new several seconds later, not tarred at all
by their experiences because they are not real. These advances can
make people desensitized to what they are reading, watching or doing
in books, films or video games, leading to a loss of control, a loss
of morality, and a loss of a grip on reality. \textit{Demonlover}
deals with these themes very convincingly, as as the rival companies'
spies infiltrate, they become accustomed to what they are seeing and
start to \textsl{enjoy} it rather than be repulsed by it, which puts
thier operation at risk, especially when in order to pull the
operation off they have to befriend each other, thus letting their
guard down enough to initially seem---and remain---convincing.\\

\textit{The Fifth Element}'s most technologically advanced scene is
that of the body reconstruction of Leeloo. This is likened in modern
times to the rise of 3D printers, which are of particular interest
given some---the RepRap, for example---can now self-replicate, which
is seen as the pinnacle of the futuristic age and will be completely
useful when, say, the world's natural resources run out and Earth, or
any planet, becomes a kind of utopia. The graphics are CGI, due to the
film's age, but still quite lifelike, and actual research (Wake Forest
School of Medicine, 2016) has dubbed ``bioprinting''---printing
tissues and organs---a real thing, so the concept was not all that
implausible to be starting to be researched in the years in which this
film would still be in the world's consciousness. Another computing
concept, but more commonly thought of as related to information
security, that this film covers is that of two-factor authentication
or, more accessibly termed, two-step verification. The principle of
this is that you authenticate---login---to for example your emails
with something you know---a password---and something you have---a code
texted to your phone number, given that nowadays people in the
Western, modern worlds, tend to always have their phones with them. I
liken this to the 5000 year deadline---something you know, as a
settler on the planet prior to its 5000 year impending
destruction---and the stones---something you have, which is brought
down by a supreme being on a spacecraft as and when you require
it---like codes being sent by text message when you log in, and to
verify that you're the real person, you have to enter them into the
login box. In the same vain, to save the planet, you need to
\textit{know} the date and \textit{possess} the stones at the right
time, in the right place. In a way, these film makers were ahead of
the technology curve in 1997 as what we have now come to know as
two-step verification did not come into common use in the modern, real
world until the mid-2000s, yet even then it was hailed as ``too little,
too late'' to protect us from the Internet's evils by world renowned
security researcher and cryptographer Bruce Schneier.\\

A big question throughout watching these films is that of why their
respective French directors focused on the technological basis of them
and making technology-related films even minimally. As I mentioned in
the introduction, France is seen as very backwards when it comes to
the use of technology, in particular computers, in most cases. The
French pioneer of telecommunications, the Minitel (of which its best
feature was its sort of ``phonebook'') was in common use until 2012
when it was retired by France Télécom in favour of the then widely
available, more modern, faster, actual broadband (or mobile) Internet
connections. (Schofield, BBC News, 2012) There exists next to no
computing education in France, with specialisms in computing only
being available at \textit{Bac Pro} level, so the inclusion of
technology in these films seems to be catered to a science fiction
loving international audience as they were both released
internationally to great acclaim. This becomes especially apparent
when one realises that their main screen language is English apart
from, in \textit{Demonlover}'s case, occasional subtitled French for
grassroots and plot effect as both of the central characters---the
rival company infiltrator women---are French.\\

In conclusion, yes, both films could be said to be both
technologically progressive and backwards (each in a different way, to
avoid a contradiction of terms), but they rely on their audiences
either not caring and watching the film for comedic or sensationalist
value. In the case of \textit{Demonlover}, the film is backwards in
some senses, as evoked above, but being ahead of its time with the
principles of biometric data collection and bioprinting to create new
beings ressembling humans, or enhanced real humans---androids or
cyborgs---artificially, and that the world will catch up with them or,
in the case of science fiction books like William Gibson's
\textit{Neuromancer}, that their work will become a world-renowned
cult classic that everyone interested will know and talk about
forevermore.

\nocite{*}

\bibliographystyle{plainnat}
\bibliography{second_assignment}

\end{document}
