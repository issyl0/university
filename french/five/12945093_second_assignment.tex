\documentclass[12pt,a4paper]{article}

\usepackage[frenchb]{babel}
\usepackage[utf8]{inputenc}
\usepackage[T1]{fontenc}
\usepackage[numbers]{natbib}
\usepackage{url}
\usepackage{breakurl}
\usepackage[breaklinks]{hyperref}
\usepackage{titling}
\usepackage{fullpage}

\setlength{\droptitle}{-10em}

\begin{document}

\title{L'imagination romanesque peut-elle nuire à notre bien-être ?}
\author{Isabell Long (12945093)}
\maketitle

Lire un livre est un passe-temps pas beaucoup apprécié dans la société de nos
jours. N'importe quel livre, mais surtout les romans. Les personnes prefèrent
lire des article de journaux, ou jouer des jeux inconséquents qui ne leur
apprennent rien, comme Candy Crush Saga. On se demande, face à ce déclin de
lecteurs, pourquoi est-ce que l'on lit, et pourquoi est-ce que ceux qui adorent
lire des livres --- qu'ils soient éléctroniques ou en papier --- l'adorent et se
retrouvent immersés dans leurs livres racontant des histoires fantastiques à
propos elles mêmes. L'imagination romanesque est une grand exemple de cette
fascination que l'on a avec les livres, et ici j'examinerai pourquoi, et si cela
est une bonne chose ou pas pour notre bien-être.\\

Penser est à la fois imaginer. Sans imaginer, on n'arrivera pas aussi bien a
penser à d'autres choses dans nos vies, à les rationaliser, ou à ouvrir notre
esprit. Il est important de penser même à des choses banales, parce qu'il est en
faisant cela que l'on pose des questions à propos de notre environnement, nos
habitudes, et beaucoup d'autres choses. Lire un livre peut être une distraction
bien nécessaire dans nos vies, parce que lire un bon livre avec une bonne
intrigue ne peut pas, généralement, ne pas nous perdre dedans pour quelques
heures de repos. En citant par exemple \textit{SciencesHumaines.com}
(2010)~\nocite{shcom} pour justifier ces dernières phrases, il est généralement
pour ces raisons là que l'on commence à lire.\\

Il y a beaucoup de genres de livres, même en précisant que des romans. Il y a
par exemple des romans policiers, des romans romantiques, des romans de
science-fiction, ou des romans qui nous amusent. Chaque genre de roman provoque
de différentes réactions dans chaque être vivant, et alors il est assez
impossible de prévoir comment une personne réagira en particulier. Dans ce sens
là, notre bien-être est remise en question, et en plus le bien-être de ceux qui
nous entourent.\\

Il est important de considérer avec cela la mode de diffusion des histoires.
Lire à haute voix nous force, quelques fois, à plus penser à si l'on lit
correctement et non pas à evoquer le sens du texte et nous « retrouver » dedans,
tant que lire seul dans notre chambre, intime, dans notre tête, fait que l'on
arrive à penser, sans peur de ne pas être correct, et alors à nous laisser nous
exprimer --- pas essentiellement à quelqu'un, mais que pour soi-même. Avec cette
pensée on pourrait même se demander si la théâtre ou le cinéma soient aussi
importants que la lecture traditionnelle à nous faire évoquer ou imaginer les
faits. Vu que ces deux méthodes de diffusion existe pour faire apprécier les
histoires à ceux ou celles qui préfèrent être visualement stimulés, ou veulent
s'informer autrement que de lire un grand texte, « l'imagination romanesque »
que traite le sujet pourra ne pas se figurer, mais les adaptations des livres ---
Harry Potter, par exemple --- nous montrent que si, mais que les adaptations
cinématiques sont une intérpretation du metteur en scène pour qu'il puisse
diriger les acteurs.\\

En quoi tout cela est difficile, et même nuisible à nos vies ? L'imprévisibilité
est une chose qui nous fait assez peur. Comment les gens vont-ils réagir ? Ils
pourront copier tout ce qu'ils lisent, et dans le cas d'un roman policier, ce ne
sera probablement pas très bien, mais on doit considérer que si les gens ont
déjà ces tendences là, il est très rare qu'ils vont commencer à lire un livre
qui a une réaction profonde --- ils savent déjà tout ce qu'ils vont faire, et
sinon, il existe désormais l'Internet, une méthode certainement plus efficace de
trouver des idées dans une crise.\\

Beaucoup de personnes confondent les romans, les livres fictifs, et la réalité
dans laquelle elles vivent, et pensent que ce qu'elles lisent reflètent leurs
vies. Il y a un danger quand on lit des romans, et cela se manifeste en oubliant
notre vie et comment elle se passe. Par exemple, si le protagoniste (héros) d'un
livre à tout fait pour lui, et ne doit faire aucun effort, le lecteur pourrait
oublier que leur vie progresse en fonction de ce qu'ils font eux-mêmes, que tout
ne vient pas sans essayer ou sans effort. Dans les cas les plus extrêmes, des
personnes peuvent perdre contact avec la réalité et s'isoler dans leurs pensées ---
et il est dans ces cas là où la pensée devient dangereuse.\\

«~Nuire à notre bien-être~» peut être interprété comme pas une chose intime,
mais une chose très grande à propos du monde. Nous ne savons pas combien
d'heures, de jours, ou d'années on a sur Terre (apart que l'espérance de vie
moyenne en France est 80 ans), et alors tout le monde est d'accord que l'on doit
faire attention à ne pas les gaspiller. On peut se demander si prendre du temps
pour s'amuser, pour se reposer, est vraiment un gaspillage de temps --- les
personnes ne peuvent pas, après tout, apprendre de nouvelles choses à toutes les
heures de la journée.\\

Le journal \textit{Psychologie}~\nocite{psychlire} écrit avec des témoignages des
psychologues que pour lire, il faut être en contact avec soi-même. Il dit même
que ne pas pouvoir lire --- penser que c'est une perte de temps, ou avoir
beaucoup d'autres choses plus pressantes a faire --- est «~souvent l’un des
premiers symptômes de la dépression~». Ne pas lire, alors, contrairement à
l'argumentation d'avant, peut nous prévenir des maladies psychologiques, alors
il ne faut pas se forcer.

\bibliographystyle{plainnat}
\bibliography{second_assignment}

\end{document}
