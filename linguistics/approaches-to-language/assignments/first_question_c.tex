\documentclass[12pt]{article}

\usepackage{fullpage}
\usepackage[numbers]{natbib}
\usepackage{url}
\usepackage{titling}

\setlength{\droptitle}{-10em}
\linespread{1.3}

\begin{document}

\title{Question C.\\
       \small{Describe the early acquisition of speech sounds by
              children: What stages do children pass through, what sounds are
              first to be acquired and what variables appear to influence the order
              of acquisition of speech sounds?}}
\author{Isabell Long (12945093)}
\maketitle

Child language acquisition is a fundamental part of the study of
linguistics and psychology alike. As its name suggests, it deals with
how those at all stages of childhood---infants through to young
adults---gain an understanding and appreciation of language.\\

There are several main stages of child language acquisition which are
mostly agreed upon. Between five and ten months old, children start
babbling.  Babbling is the child's repetition, potentially an attempt
at imitation, of sounds they hear around them. Examples of this
include ``ba ba ba'', ``ma ma ma'' or ``goo goo''---comparative
nonsense to us adults, but a ray of hope for those who may worry that
their child suffers from a developmental disorder. Indeed, an absence
of babbling at this age can indicate cause for concern, i.e.\ the
potential onset of deafness---children can't articulate sounds if they
don't hear them!---or determining that later in life the child may
require speech therapy. Some parents may get excited about their
children saying ``ma ma ma'' because they think that he or she is
saying ``mama'', thinking that they are imitating the motherese (baby
talk) that they would have said to their children to get their
attention and help them learn faster~\cite{baby-talk-helps-learning}.
This excitement is premature, however, as children at this young an
age cannot properly form words.\\

It is commonly accepted that expression of first words happens at what
is called the variegated stage between ten and eighteen months. In
this stage, vowels are acquiried earlier than consonants, yet /b/
and /p/ are the most common consonants found in a developing child's
repertoire~(\citeauthor{ogrady-cla}, \citeyear{ogrady-cla}), and
research on children who are learning to speak English has found that
nouns are learned and used before verbs~(\citeauthor{nouns-before-verbs},
\citeyear{nouns-before-verbs}). For example, ``pick me up'' would
simply be articulated as ``up'' because ``pick'' is a verb (``to
pick''), and ``me'' as the subject of the verb is
unimportant---parents or other people such as nannies who have
frequent contact with children at this stage would know what they want
and fulfil their wishes by picking them up. In schools, singing is
frequently used as a language-learning device. There is strong
evidence to suggest that language acquisition is sped up by singing,
hence the existence of the ABC song for the alphabet. Children
map a mental model of song lyrics more easily than if they were to
hear a poem read out in a boring voice, because---in hearing people at
least---our brains are inherently wired for
sound.~\cite{wired-for-sound}\\

Even before the earliest speech development phase agreed upon, 6--8
months, researchers have found that talking to the baby in the womb
(\textit{in utero}) is beneficial for its linguistic development. In a
study by \citeauthor{talking-in-utero} in \citeyear{talking-in-utero}
at the University of South Carolina, it was determined that babies who
were read to while in the womb were better at recognising their
mothers' voices and the contents of the book, compared to those
foetuses who had not been explicitly communicated with in this
structured way. Therefore, it has been found that language development
happens in an unstructured way before the mainly recognised stages,
because babies once born can recognise their mothers' or fathers'
voices and intonations above anyone else's.\\

In cases of neglect or abuse (both of which are sadly found together),
from birth or up until the child is older, children who are not
communicated with at all can have severe difficulties not only
emotionally, but in acquiring and using even the simplest of language
or grammar. This manifested itself strongly in a
\citeyear{effects-neglect-language} study conducted by
\citeauthor{effects-neglect-language} which not only examined spoken
language ability, but also the degree to which they understood through
listening. It was found that neglect of this sample of children does
indeed make both their spoken and auditory abilities weaker. This
study does nothing to prove or disprove the critical period
hypothesis: that there is a key language-learning age and if that
window is missed, severe language-learning difficulties can arise. For
that, we must turn to research such as the famous case of a child
named Genie, who, when found aged thirteen, had no language skills
whatsoever and did not acquire more than basic knowledge even after
specialist teaching.~\cite{genie} This leads us to agree with the
critical period hypothesis as Genie at thirteen with no interaction
was brought up as a ``feral'' child. However, science would not be
accountable without multiple examples. Let us discover another
linguistics poster-child---`Isabelle`---who was deaf and found
isolated in a dark room at the age of six. She was also mute as she
had had very little interaction with anyone else.  Conversely, with
specialist training Isabelle was able to acquire language at a very
fast rate compared to the then-normal language ability of six year
olds.~\cite{isabelle} The studies however did not take into account
that she was deaf, which, according to~\citeauthor{isabelle-deaf},
could have severely affected her abilities, although said abilities
were extremely positive and advanced her peers. That said, despite
some flaws such as research not always being conducted by a linguist,
both of these cases prove the critical period hypothesis: six years is
still early enough, yet thirteen years \textit{can} be too late.\\

Research is mostly concentrated on first language acquisition, but
second language acquisition in the case of those with bilingual
parents, or if the child lives in a country where those around him or
her don't speak the same language as their parents, is also
interesting and has its place. This has been studied by the likes of
\citeauthor{important-second-language-acquisition} in
\citeyear{important-second-language-acquisition}. For example, those
who learn a second language at a young age have more of a chance of
developing a native or near-native accent as found in Korean students
learning English by \citeauthor{accent}~(\citeyear{accent}): those who
had arrived in the United States earlier in their lives had a less
noticeable foreign accent compared to those had arrived 23 years
old.\\

Overall, it can be seen from the research that children acquire sounds
at different stages and with varying degrees of success depending on
their circumstances and upbringing.

\bibliographystyle{plainnat}
\bibliography{question_c}

\end{document}
