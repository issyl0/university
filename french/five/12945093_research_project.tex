\documentclass[12pt]{article}

\usepackage[numbers]{natbib}
\usepackage[frenchb]{babel}
\usepackage[utf8]{inputenc}
\usepackage[T1]{fontenc}
\usepackage{url}
\usepackage{fullpage}
\usepackage{breakurl}
\usepackage[breaklinks]{hyperref}
\usepackage{titling}

\setlength{\droptitle}{-10em}

\begin{document}

\title{L'immigration en France et en Angleterre: une montée de l'exil?}
\author{Isabell Long (12945093)}
\maketitle

% Introduire le sujet.
L'immigration est un phénomène très important de nos jours. Dans ce
travail, je tenterai d'explorer la raison pour laquelle le taux
d'immigration en France à augmenté dans les années 2000,
particulièrement le taux d'immigration des Anglais. Bien que ce soit
``à la mode'' en ce moment, ce travail n'abordera pas la situation des
réfugiés des pays d'Afrique, mais que des personnes qui
\textit{veulent} déménager et non pas celles qui \textit{doivent} pour
fuire leurs pays d'origine à cause de persecution.\\

% Definir le sujet.
Le verbe «~immigrer~» est defini par «~[v]enir se fixer dans un pays
étranger au sien,~» ainsi que le mot «~émigrer~» est defini par
«~[q]uitter son pays pour aller se fixer dans un autre.~» (Larousse,
2016) Cet différence est très important, parce que souvent les médias
confondent l'immigration et l'émigration, et ainsi produisent des
statistiques plus élevées qu'en réalité qui choquent leur public et
souvent permet aux spectateurs d'avoir une idée plus négative du
sujet. Comme j'ai écrit dans l'introduction, je ne parlerai pas
d'immigration ou émigration forcé, mais immigration ou émigration
choisi. Ainsi, le mot «~migration~» signifie les deux en un seul
mot.\\

Durant mes recherches, je voulais envoyer un sondage à des
personnes en leur demandant pourquoi elles avaient migrées, mais je
n'avais pas assez de temps, alors les données quantitatifs et
qualitatives réspectivement viennent des recherches des autres (at, pas
les miennes. Pourtant, il existe beaucoup de données intéressantes.\\

% Pourquoi déménage-t-on ?
Les personnes déménagent pour beaucoup de raisons. Entre autres
(j'aborderai ces autres plus tard), elles sont: l'envie de connaître
d'autres cultures, vouloir apprendre d'autres langues ou perfectionner
celles que l'on sait déjà, ou suivre notre famille (par exemple, nos
très proches) qui voyagent eux-mêmes. Bien sûr, ces dernières trois
choses peuvent être faits simplement en vacances, et seront peut-être
mieux faits en vacances, non pas en bousculent toute une vie (ou
quelques vies, si d'autres personnes sont impliquées comme par exemple
des enfants ou des amoureux). Les personnes peuvent déménager pour
mener une vie plus efficace, ou une vie plus enrichissante. Bien sûr,
mener une vie plus efficace et enrichissante s'effectue différament
en fonction de chaque personne. La plupart des Anglais déménagent pour
disparaître de leur vie quotidienne, pour arrêter de courir partout
faisant ce que la société moderne leur demande.\\

% La fonction de l'Union Éuropéene sur la migration --- plus facile,
% plus de lois en commun, plus de protection si les choses ne vont pas
% comme l'on imagine. L'asile. Une réponse commune à toute l'UE.
Le fait que la France et l'Angleterre appartiennent (pour le moment,
dans le cas de l'Angleterre !) à l'Union Européene a un grand
influence sur la migration des deux côtés. Il existe plusieurs
avantages d'être membre, particulièrement concernant la migration. Les
personnes, qu'elles soient natif au pays en question, en visite, ou
des immigrées, il existe des garanties de sécurité, des lois communs à
tous les pays, et les ambassades de chaque pays dans les autres pays
où les personnes peuvent visiter pour, par exemple, renouveler leurs
passeports ou cartes d'identité. Parlant de cartes d'identité m'amène
bien au fait qu'en Europe il n'y a pas censé être des frontières, les
pays partent sur le principe que tout mouvement est autorisé, alors
les personnes peuvent conduire une voiture depius la France jusqu'en
Belgique, par exemple, sans contrôles, ce qui est important pour la
liberté et pour encourager l'immigration. Contrairement à cela, et
pour nous distancer du débat sur l'Union Éuropéene qui à l'impression
d'être partout en Angleterre en ce moment, chaque personne qui visite
un pays en Europe profite du Déclaration Universelle des Droits de
l'Homme, ce qui leur garantie la protection contre, par exemple, la
peine de mort, leur dignité, leur égalité avec les autres, et la
liberté d'expression. Ceci est une réponse commun sortant de chaque
pays en Europe en fonction de son géographie, qu'il soit membre de
l'Union Éuropéene ou pas.\\

Étant mécontents des mouvements différents en Angleterre, beaucoup de
personnes ont déménagés pour échapper ses changements (autrement dit
évolutions). Ceci est intéressant parce que les choses que les
gens n'aiment pas sont régulièrement réligieuses (les musulmans venant
en Angleterre imposer leurs lois, comme disent beaucoup de personnes
qui ont malheureusement une politique de droite qui empêche la liberté
d'expression), et la France, officiellement en ce qui concerne la
politique et la vie « normale » ne l'est pas. Cela m'amène au
conclusion que les personnes aiment la France parce qu'elles ne seront
pas jugées en fonction de leur religion, parce qu'il y a une séparation
entre l'État et l'Église, connu bien sûr comme la laïcité. La laïcité
rend l'État impartiel, pas réactif en fonction de ce que les personnes
croient, parce qu'elles ne peuvent pas questionner cela quand ils
traitent d'une personne ou d'un cas. Ainsi, l'asile est (en théorie)
plus facile à demander parce que ceux qui le jugent n'ont pas le droit
d'être dissuadés (et alors discriminer) en fonction du réligion du
demandeur d'asile.\\

% Les immigrants souffrent-t-ils avec les changements ?
Aussi, on peut se demander si les migrants souffrent à cause des
changements. Même s'ils ont décidé de quitter leur pays principal et
aller dans un autre, ils font cela avec un peu de malaise, un peu de
peur, parce que déménager, quitter sa famille et ses amis et, même si
l'on a de l'argent, trouver un logement, toutes ces choses sont
toujours difficiles. Comme le site du gouvernement Anglais nous
rappelle: ``France is our closest neighbour, but life in France can be
very different'' (« La France est notre voisin le plus proche, mais la
vie peut être très différent »). Prenant l'exemple des Anglais allant
en France, ils y aillent souvent pour s'échapper de leurs vies
quotidiennes, des routines qu'ils ont obtenus au fil des
années---comme avec beaucoup d'autres nationalités---et veulent être
isolés. Mais, être isolé en France, particulièrement dans la campagne,
par exemple, n'est pas comme être isolé en campagne en Angleterre: là,
il existe toujours es magasins ouvert les midis et les dimanches, tant
qu'en France, non---les Français sont très strictes et suivent leurs
traditions.  Bien sûr, aller de la France à l'Angleterre, ou de
l'Angleterre à la France comporte un changement de langue. Les Anglais
apprennent le français, et les Français apprenent l'anglais à l'école,
mais quand ils déménagent, souvent ils ne savent pas plus que ce
qu'ils ont appris à l'époque quand ils avaient douze ans. Ceci rend la
communication difficile, parce que même si les personnes avec qui ils
dialoguent en France parlent l'Anglais, ils seront peut-être réticent
de le parler, pensant (comme les Anglais pensent avec les étrangers
venant en Angleterre) que les immigrants (ou visiteurs) doivent
apprendre la langue officielle du pays dans lequel ils sont. Mais,
sans parler la langue, sans pratiquer, on n'avance pas et alors voilà
un cercle viscieux. Soit cela, soit on reste dans des communautés
principalement composé des personnes de notre pays natal, et on
n'essaye pas de s'intégrer, ou s'intègre qu'un minimum. Ceci se montre
dans les communautés en Bretagne---très proche de
l'Angleterre---consacré presque totalement d'Anglais, et concernant
les Français, beaucoup d'eux se sont concentrés à Londres, au tour de
South Kensington à Londres, où il y a une école française. Cette
compartimentation aide la migration parce que ceux qui voyagent ont,
disons, un pied de terre dans leur pays de destination. Par contre,
certains demandent si cela sagit de souffrance, si l'on se sépare et
même pourquoi est-ce que l'on partent à un autre pays pour ne pas
s'intégrer dans la culture. Mais, revenant à la souffrance, comme tant
de personnes ont migrées vers soit la France ou l'Angleterre, beaucoup
d'entre elles le font pour améliorer leur vie. Si elles ont des
enfants, la famille déménage pour améliorer la vie des enfants, pour
leur donner d'autres opportunités qu'ils n'auraient pas eus, d'autres
expériences, d'autres amis. Les enfants souffrent peut-être plus que
leurs parents, mais des études psychologiques ont montré que les
cerveaux des enfants sont plus « plastifiés », plus ouverts aux
changements, plus adaptables, et alors les parents espèrent que leurs
enfants ne sortent pas de leur enfance en ne pas apprécient les
opportunités, même si les changements ont été difficiles parce que
déménager les a éloigné des amis et des cadres qu'ils connaissaient.\\

% Histoire de l'immigration - voir une montée grace aux chiffres.
Certains disent que la migration est un phénomène récent,
certainement dans la quantité que l'on voit dans les années 2000.
Mais, la migration a toujours existée. En quittant la France et en
parlant d'émigration partout dans le monde pour élargir nos horizons
un moment, les chiffres de l'année 2000 d'immigration en Angleterre
étaient 479 000, soit 23 000 venant de France. Il faut comparer cela
avec les chiffres de 2014, les plus récents que je n'ai pu trouver. En
comparant, alors, il y a eu 632 000 personnes qui ont immigrées en
Angleterre, dont la France ne figure pas sur la liste des trois pays
les plus communs d'où ils sont venus : ceux-là étaient l'Inde, la
Chine et la Roumanie.~(Office of National Statistics, 2014)\\

Pour ne pas avoir toujours une vue moderne sur la migration, il est
important de considérer et analyser la migration durant les années du
XIXème siècle. Pour nous aider avec cela concernant simplement la
France, philosophe et anthropologue Guy le Moine a écrit
\textit{L'Immigration en France} en 1984 qui traverse les années 1851
à « nos jours », disons 1984. Les premiers immigrants étaient les
Belges et les Italiens, dont les personnes des pays voisins à la
France, dont les mouvements étaient très facile, sans problèmes et
bureaucratie grâce au fait qu'aucun pays n'était membre de l'Union
Européene parce qu'elle n'éxistait pas jusqu'en 1993. Pour essayer de
contrôler l'immigration quand les chiffres ont explosé en 1946, le
gouvernement a interdit tout immigration (pp.\ 8--9), mais cela n'a
pas dissuadé la population: presque deux millions de personnes ont
immigrées en France cette année là. Il existe plusiers raisons pour
lesquelles la France aurait pu imposer des restrictions. Surement,
1946 était juste après la guerre et elle a peut-être voulu se
récuperer, se reposer, se remettre après et ne pas avoir un grand
influx de personnes qui avaient pour but d'échapper de leurs pays
d'origine grâce à la mauvaise économie, les maladies dû à la manque de
nourriture ou de sanitation, ou quoique ce soit. L'économie est aussi
pourquoi une grande partie de la population émigrent ailleurs. Si les
personnes ont gagné une grande somme d'argent, elles veulent surement
le dépenser où cela va leur donner plus d'opportunités dans le futur.
Si ce n'est pas dans leur pays principal à cause de la récession
économique, alors ces personnes là vont regarder ailleurs, ainsi,
s'ils achètent dans un autre pays, motivant et stimulant l'économie
de cet autre pays.\\

% Conclusion.
Pour conclure, oui, le taux de migration à augmenté partout dans le
monde, augmentant ainsi l'éxil des personnes depuis la France et
depuis l'Angleterre. Cela se fait pour plusieurs raisons, mais les
plus communs sont l'économie, la famille, fuir des conflits comme la
guerre, la politique avec laquelle les personnes ne sont pas d'accord,
la qualité de vie, les études, élargir les horizons et ouvrir les yeux
à de différents cultures. Cet étude à été très intéressant et m'a
ouvert les yeux à la souffrance des migrants, mais aussi les bienfaits
de migration pour bousculer le « normal » des pays modernes, pour ne
pas qu'ils deviennent stagnants.

\nocite{*}

\bibliographystyle{plainnat}
\bibliography{research_project}

\end{document}
