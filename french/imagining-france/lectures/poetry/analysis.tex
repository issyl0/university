\documentclass{article}

\usepackage{fullpage}

\begin{document}

\section*{Imagining France, lectures on 16th century poetry and philosophy, semester two (2015)}

\subsection*{Montaigne, Essais}

``Essais`` --- not as we know them --- a series of attempts at
judgement and to work out how to live --- inconclusive, always moving,
always ongoing. The subject of these Essais is himself --- he writes
about what he knows best.\\

1562--1592 $=$ various religious wars, between Catholics and
Protestants.\\

Montaigne's use of diagonal strokes, or [A], [B], [C]. Montaigne's
works have various layers:

\begin{itemize}
  \item{1580 -- \slash -- A.}
  \item{1588 -- \slash\slash -- B.}
  \item{1595 -- \slash\slash\slash -- C -- produced posthumously by Marie de
        Gournay, his adopted literary daughter.}
\end{itemize}

\subsubsection*{L'Exercitation}

(p.\ 43 of the French translation)\\

Montaigne suffers an accident on horseback and reflects on his own
fear of death. He lived in a very dangerous area at the time, near
Bordeaux. He was twice major of Bordeaux. He was alone on horseback
--- he did not take any servants with him --- but it was a
battle-hardened horse. He lacks historical specificity for this
incident, not being able to remember when exactly it happened. He
recounts how imposing and ugly the horse's mouth was.\\

Montaigne shows his feeble side by focusing on his ``small`` self and
``small`` horse compared to the huge wars going on.\\

The French text contains Latin quotations between paragraphs. These
give different ideas on life and its meaning. Some of them Montaigne
doesn't agree with compared to what he's saying, but they lend weight
and authority. These reinforce the breadth of his reading and study.
The reader gets the impression of continuity of thoughts and ideas in
humans --- Greek or Latin contains lots of things that strike chords
even in this modern world, emotionally.\\

Montaigne is often trying to achieve a sort of ``stream of
consciousness`` account of events, using very long sentences with lots
happening in them. He mentions that this accident is the only time he
has lost consciousness, as if it shows \textit{weakness}.\\

He mentions the body as well as the mind and soul. ``\textbf{The
feeling of physical pleasure must involve the mind too}``, so it
doesn't disappear and instead intensifies --- \textit{plaisir
intellectuellement sensible}. His graphic description of his journey
back home after his accident, where he fills a bucket with blood,
emphasises this importance.\\

Montaigne en Mouvement -- Franco--Swiss critic Jean Starobinski:
Montaigne slides from \textit{douleur} to \textit{douceur}.  This is
an example of \textbf{paronomasia}. Montaigne realises that death is
not all that horrible or terrifying and can be easily slipped into.\\

In the 1588 edition of this essay, he opens up his bank of quotations
with an Italian quotation, to show his breadth of knowledge once again
and how it has increased over the years between publications.\\

Montaigne is shown to love Latin poetry through copious numbers of
Latin quotations such as ``ut tandem sensus convalwere mei''.
Lucretius wrote in Latin on suicide and the nature of the universe:
``De Rerum Natura''. Socrates said that ``I know that I know
nothing``.\\

Montaigne went through three phases concerning what he thought about
death:
\begin{itemize}
  \item{Stoicism.}
  \item{Skepticism.}
  \item{Epicurean.}
\end{itemize}

The long phrase inside the parentheses (``car je l'avoy veu\ldots'')
changes tenses many times which evokes his thought processes and his
sudden realisation that he was nearing death. He realises that death
is a neighbour and not to be feared.\\

Montaigne uses real-life events (buckets of blood, falling off horses
etc.) to explain usually abstract philosophical points of view, such
as death.\\

Towards the end of the volume (page 50): Montaigne loved horse riding
and moving. In this chapter, he presents himself as a ``noble
d'\'ep\'ee'' rather than a ``noble de robe''.\\

\end{document}
